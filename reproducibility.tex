\chapter{Recommendations for utilizing and reporting population genetic analyses: The 
reproducibility of genetic clustering using the program \textsc{structure}}
\label{chap:reproducibility}

%% \section{Summary}

%%Reproducibility is the benchmark for results and conclusions drawn from scientific studies, but systematic studies on the reproducibility of scientific results are surprisingly rare. Moreover, many modern statistical methods make use of `random walk` model fitting procedures, and these are inherently stochastic in their output. Does the combination of these statistical procedures and current standards of data archiving and method reporting permit the reproduction of the authors' results? To test this, we reanalysed data sets gathered from papers using the software package \textsc{structure} to identify genetically similar clusters of individuals. We find that reproducing \textsc{structure} results can be difficult despite the straightforward requirements of the program. Our results indicate that 30\% of analyses were unable to reproduce the same number of population clusters. To improve this, we make recommendations for future use of the software and for reporting \textsc{structure} analyses and results in published works.

\section{Introduction}
The reproducibility of scientific research is fundamental to maintaining scientific rigor 
and advancing science \citep{Price:2011}. Full experimental replication provides the most thorough 
means of verifying published empirical results, but this approach can be impractical due to 
the difficulty in obtaining identical samples and large financial and time commitments 
\citep{Peng:2011}. Diminishing costs and advancing technology have resulted in a plethora of large 
genetic data sets, while at the same time, there has been an increase in the complexity of 
software applications. A previous investigation into the reproducibility of microarray studies 
found that few were fully repeatable, as many suffered from ambiguity in the methods, discrepancy 
in the results, and lack of available data or software \citep{Ioannidis:2009}. Maintaining 
the rigor of today's scientific research may therefore prove a more difficult task than expected, 
as both the empirical results and the often complex analyses need to be reproducible. Efforts to 
encourage and implement data archiving and sharing are expanding, and these create the opportunity 
to test the validity and reproducibility of scientific results \citep{Whitlock:2010}.

Reproducing results within the field of molecular ecology is especially difficult because 
biological samples are unique to their particular place and time, and subsequent samples may 
reflect different ecological or evolutionary forces \citep{Wolkovich:2012}. Researchers 
therefore tend to test the same overarching hypothesis with samples from different taxa and 
locations, in the hope of arriving at a more general and repeatable pattern. However, drawing 
broad conclusions from the results of many studies is ineffective when the results of the 
individual studies cannot be reproduced from their underlying data. It is thus essential 
to test the reproducibility of statistical analyses at the level of individual papers as well. 
To examine how well we could recreate the results from typical molecular ecology studies, we 
investigate, as an example, the reproducibility of studies that used genotype data to identify 
genetically similar clusters of individuals with \textsc{structure} \citep{Pritchard:2000}. Many studies 
use clustering results based on \textsc{structure} to perform further analyses, making it an important 
foundation upon which inferences are built. We ask whether (i) archived data sets are sufficiently 
complete and well annotated that they can be reused, (ii) published articles specify all the 
methodological details necessary to reproduce the analysis, and (iii) where possible, the same 
conclusions can be reached by reanalysing the archived data. Although reproducibility has many 
different aspects, we use it here to mean the agreement between results obtained through analysing 
identical data sets using the same analytical method but under different conditions (different observers, 
computers and starting points in computer algorithms). We reanalysed 23 articles from 2011 that used 
\textsc{structure} to infer genetic clustering and also checked the level of data completeness and methodology 
reporting in an additional 37 articles.

\section{Methods}
\subsection{Obtaining Datasets}
We gathered \textsc{structure} data sets associated with 23 papers published in 2011: 21 from 
Molecular Ecology, and two from the journal PLoS One. Data were obtained from the online data 
repository Dryad \citep{Dryad} in November 2011, NCBI GenBank, or from the 
supplementary material accompanying the paper. With one exception, we excluded papers where 
data were archived on GenBank due to the difficulty of compiling individual accessions into 
the correct format for \textsc{structure}.

For a broader assessment of data set completeness and methods reporting, we also included 37 
data sets obtained by contacting the authors of original research papers published in PLoS One, 
PLoS Genetics and BMC Evolutionary Biology and collected as part of a separate study (T.H. Vines et al., unpublished data).

\subsection{The Program \textsc{structure}}
The freely available Bayesian clustering program \textsc{structure} \citep{Pritchard:2000} 
is the most commonly used application to infer population structure, with over 5000 citations 
in Web of Science as of June 2012. \textsc{structure} uses multilocus genotype data to 
describe and visualize population structure based on allele frequencies of the data.

\textsc{structure} is capable of analysing a variety of genotype data, including both 
codominant markers (microsatellite and single nucleotide polymorphism, SNP; \citealt{Pritchard:2000}) 
and dominant markers (amplified fragment length polymorphism, AFLP; \citealt{Falush:2003}). 
The model uses Markov chain Monte Carlo (MCMC) simulations to estimate the group membership 
of each individual, assuming Hardy-Weinberg and linkage equilibrium within groups, random 
mating within populations and free recombination between loci \citep{Pritchard:2000}. 
Due to the random walk characteristic of the MCMC methods, \textsc{structure} outputs 
are not expected to produce identical results, yet the approach should be robust enough 
to yield identical conclusions when reproduced. The program is initiated with a required 
text file containing individual genotype data and labels as well as optional information 
on population assignment, sampling sites or locus names. The user specifies several 
essential parameters regarding the ancestry model, the allele frequencies model, 
the length of the burnin (initial runs of the simulation during which data are not 
retained to ensure results are not dependent on initial conditions), length of run 
time (number of MCMC repetitions during which data are retained), the number of 
independent replicates of each set of parameters and the range of number of clusters 
(\emph{K} values) to be tested. These can be specified directly in the graphical user interface 
or in a separate text file when run in the command line. In addition, it is possible to 
specify extra parameters, mainly regarding the Markov chain process as well as a sampling 
location prior. The details of the model are described by \citet{Pritchard:2000, Pritchard:2007} 
and \citet{Falush:2003, Falush:2007}.

\textsc{structure} outputs are typically analysed to infer the optimal \emph{K} by one 
or a combination of methods. In the method described in \citet{Pritchard:2000}, the 
optimal \emph{K} is chosen by plotting the log probability of the data (referred to as 
$\ln \Pr(X|K)$ in \textsc{structure}'s manual, \citealt{Pritchard:2007}) against a range of \emph{K} 
values and selecting the \emph{K} with the highest $\ln \Pr(X|K)$ or the one after which the 
trend plateaus, while also taking into account the consistency of the groupings across 
multiple runs with the same \emph{K}. An alternative method, described by \citet{Evanno:2005}, 
formalizes an ad hoc approach based on plotting the secondorder rate of change in $\ln \Pr(X|K)$ 
for successive \emph{K}s (referred to as $\Delta K$) against a range of \emph{K} values, and selecting 
the true \emph{K} based on where the maximal value of this distribution occurs. As 
emphasized in the \textsc{structure} manual \citep{Pritchard:2007}, selecting the optimal 
\emph{K} can be quite a subjective procedure and is best inferred when the biology and 
history of the organism are taken into account. Replicate \textsc{structure} runs can be 
combined using the software programs \textsc{clumpp} \citep{Jakobsson:2007} and 
\textsc{structure harvester} \citep{Earl:2012}. Bar plots depicting the ancestry 
proportions (or membership coefficients, \emph{Q}) of individuals in each cluster can then be 
created, for example with the software \textsc{distruct} \citep{Rosenberg:2004}, to visualize the population clusters.

\subsection{Analysing Data Sets}
We followed procedures for analysis as described in the methods section of each publication 
and used default settings for parameters that were not specified. Several publications 
performed multiple \textsc{structure} analyses, which we counted independently for a total 
of 34 analyses. We made use of the Bioportal computing resource (https://www.bioportal.uio.no/; \citealt{Kumar:2009}) 
or local desktop computers. Output was compiled with \textsc{structure harvester} and processed 
following the authors' description, including CLUMPP analysis where appropriate. We first assessed 
whether we could reproduce the \emph{K} values from the original study based on the methods used by 
the authors. Then, whenever possible, membership coefficient bar plots were visually compared by 
multiple authors of the present study to assess whether our results were a true reproduction of 
the original results. When we concluded the same value of \emph{K} as the authors, we deemed the 
analysis as reproduced unless the membership coefficient bar plots showed strikingly different results.

For the broader survey of data set completeness, we evaluated whether the sample size and number of 
loci described in the paper matched the obtained data set from both the data sets obtained from online 
material (23 studies) and email correspondence (37 studies). To check the overall standards for 
reporting parameter settings within either the methods section or in a supplemental file, we also 
recorded the number of `essential` parameter settings (range of \emph{K} values tested, length of 
burn-in, length of MCMC repetitions, number of independent replicates, the admixture model and 
allele frequencies model) given in the paper or supplemental material for all of the above analyses.

\section{Results}
Of the 23 papers, we attempted to reanalyse using data from supplementary materials or online 
repositories, two papers did not have archived data present at the time, making their reanalysis 
impossible. Three papers (13\%) provided data where the number of individuals and/or loci specified 
in the publication did not match those present in the data set, or the authors performed their 
\textsc{structure} analysis on an unspecified subset of the archived data. Of these 23 papers, 
three selected \emph{K} using the Pritchard method \citep{Pritchard:2000}, seven used the Evanno 
method \citep{Evanno:2005}, eight used a combination, one used a nonparametric Wilcoxon test, 
two did not specify their method, one used no standard method and rather utilized \emph{K} = 2 to 
identify hybrid individuals and one discussed a comparison of two \emph{K} values obtained in a 
previous study. We therefore also did not assess the reproducibility of these final two papers, 
leaving 19 papers (containing 30 analyses) that we attempted to reproduce. See Tables \ref{tab:repro_orig} and \ref{tab:repro_new} for full characteristics of all analyses.

We were able to reproduce the authors' inference of \emph{K} for 70\% (21 of 30) of the analyses 
(Figure \ref{fig:repro-1}). All of the successfully reproduced data sets consisted of microsatellite 
genotypes. In general, microsatellite data sets were analysed using longer burn-in and MCMC run 
lengths as well as more independent replicates; however, there was no significant difference in 
an overall proxy for run length ([length of burn-in + length of MCMC repetitions] 9 number of independent replicates) between 
analyses that were reproduced and those that were not (t = 0.0617, d.f. = 13.564, P-value = 0.95). 
Comparing these parameters individually, we found a trend of longer burn-in (t = 1.8706, d.f. = 26.991, P-value = 0.072) 
but not of more MCMC repetitions (t = 1.6537, d.f. = 21.677, P-value = 0.11) or an increase in the 
number of independent replicates (t = 1.1442, d.f. = 7.511, P-value = 0.29) for reproduced studies. 
Comparison of the \emph{K} values chosen by the original authors versus our reanalysed \emph{K} results 
showed a significant correlation of 0.5934 (t = 3.9703, d.f. = 29, Pvalue = 0.0004; Figure \ref{fig:repro-2}).

\begin{figure}[]
\centering
\makebox[\textwidth]{
        \includegraphics[width=1.0\linewidth]{Figures/repro_fig1.pdf}}
\caption[~- Results of \textsc{structure} reanalyses.]{Results of \textsc{structure} reanalyses. Initial branching arrows show the numbers of analyses 
resulting in different outcomes at the point of selecting a \emph{K} value. The subsequent arrows show 
the numbers of analyses successfully reaching the point of matching membership coefficients. Size of 
arrowheads is proportional to number of analyses present. 
*When \emph{K} was not inferred, we attempted to match membership coefficients across all \emph{K} values (still only counted as 1 analysis). 
**When \emph{K} was not reproduced, we compared membership coefficients at the authors' chosen \emph{K}. 
***For incomplete data, analyses could not be run.}
\label{fig:repro-1}
\end{figure}

We also assessed the completeness and description of all 60 data sets that we obtained and found 
35\% to be either incorrectly or insufficiently described by the authors. We found that 17 data 
sets did not match the description given in the paper, most typically because the data contained 
a different number of loci or individuals than suggested by the paper. Lastly, four papers did not 
give any clear description of the number of individuals, loci or both used in their \textsc{structure} 
analysis, making it impossible to judge how well the archived data matched the data analysed by the authors.

Authors' descriptions of the essential parameters used to run \textsc{structure} varied markedly, 
ranging from 0 described parameters to a maximum of 6 (median = 6). We found a significant difference 
in number of essential parameters between two of the journals (t = 3.31, d. f. = 40.27, P-value = 0.015), 
with Molecular Ecology having a mean of 5.7 parameters specified and PLoS One 4.6 (PLoS Genetics, 4.7 and 
BMC Evol. Biol., 4.8). Overall, length of burn-in ranged from 1000 to 50 000 000 (median = 50 000), 
while MCMC repetitions ranged from 10 000 to 500 000 000 (median = 450 000). Independent replicates 
ranged from 3 to 100 (median = 10).

\begin{figure}[]
\centering
\makebox[\textwidth]{
        \includegraphics[width=1.0\linewidth]{Figures/repro_fig2.pdf}}
\caption[~- Comparison of \emph{K} values for original studies versus reproduced studies.]{Comparison of \emph{K} values for original studies versus reproduced studies. Dotted line 
indicated the 1:1 line, points are jittered for better visualization.}
\label{fig:repro-2}
\end{figure}

\section{Discussion and Recommendations}
Reproducibility is a foundation of scientific research. The widespread application of \textsc{structure} 
makes it an ideal case study to test the ability to reproduce molecular ecology results that rely on large 
data sets and complex algorithms. As \textsc{structure} results often serve as the underpinnings for 
further analyses and conclusions within a study, it is important to assess whether the implementation 
of the program, subsequent analysis and associated conclusions are properly reported and can be reproduced.

We find that reproduction of \textsc{structure} results can be difficult to achieve, despite the 
straightforward input requirements of the program (a genotype file and two parameter files). Our 
results show that 30\% of analyses are not reproducible. A large factor in the failure to reproduce 
these analyses was the availability of data in a form that could be readily understood by researchers 
not familiar with the study system. Had we included studies with no data available as the starting 
point for our reanalysis, our assessment of failure to reproduce would have been even higher, particularly 
for journals without a strongly enforced data archiving policy (see \citealt{Vines:2013} for further discussion of data accessibility).

We recognize that assessing reproducibility is inherently difficult. Our main evaluation criterion 
(same \emph{K} value) is only a small part of full reproducibility of these studies, but the most 
objective one. Furthermore, it is difficult to disentangle the nonreproducibility caused by the 
stochastic nature of the program from that caused by both discrepancies in data sets available versus 
those used by authors and their reported methods. The trend of longer burn-in lengths in reproduced 
studies suggests that at least a portion of the poor reproducibility of some studies is due to the 
inherent stochasticity of the Monte Carlo approach itself. In at least one case, we can attribute 
our failure to reproduce the study to insufficiently described, complex analyses performed; however, 
there seemed to be no other outstanding characteristics of nonreproducible studies. It is important 
to note that although \textsc{structure} is the most commonly used program, in some instances, other methods
may be more appropriate for a given data set. For example, performing a PCA allows examination of 
variability within clusters, other Bayesian methods such as the program \textsc{instruct} \citep{Gao-etal:2007} 
allows inbred genotypes to be used, \textsc{tess} \citep{Chen:2007} utilizes spatial information, 
and \textsc{baps} \citep{Corander:2003, Corander:2004, Corander:2006, CoranderMarttinen:2006} aids in detection 
of admixed individuals. Using the right program is not only essential to drawing correct conclusions, 
but may also improve reproducibility of results. Further discussion of additional approaches can be 
found in \citet{Latch:2006} and \citet{Franccois:2010}.

In addition, we may have judged a study to be nonreproducible despite differences in the final results 
that may or may not have biological significance. The correlation between original and reproduced 
\emph{K} values implicates this, yet there is still clearly room for improvement. With such widespread 
use within its field, it is important that users of \textsc{structure} properly implement the software, 
regardless of whether or not they possess a full understanding of the algorithm underlying the analysis. 
To ensure that published results can be reproduced, we make the following recommendations for future users 
of the program. Although our study is specific to \textsc{structure}, many of these recommendations are 
applicable to other types of analysis.

\begin{itemize}
\item[1] For archiving purposes, authors should be encouraged to provide the final version of both the genotype 
and parameter files. We propose that authors archive genotype data from all individuals. If only a subset 
was used in the analysis, these individuals should be clearly identified in the same file so that this 
information is retained. The parameter files include all the settings used in the analysis, hence 
archiving the entire file avoids any confusion regarding use of default settings when not explicitly 
stated by the authors. When using the graphical user interface version of the software, the parameters 
can be exported from the program in text format for archiving purposes.

\item[2] Authors should ensure that burn-in and run lengths are sufficient. We found remarkable variation in 
parameters affecting the computational demands of the analysis (burn-in time, MCMC repetitions, and 
replicate runs). Though we found no significant difference between an overall proxy for run length and 
reproducibility and only a slight trend individually for burn-in time, given the advances in computing 
power, we feel that the proposed minimum requirements, dating back to the software's advent more than a 
decade ago, should be increased. It is difficult to set a standard, as variability across data sets in 
the number of loci, their levels of polymorphism, and the amount of population structure present all 
also contribute to the program's ability to successfully detect the appropriate \emph{K} 
\citep{Rosenberg:2001, Latch:2006, Gao:2007}. We would advise a minimum of at 
least 100 000 burn-in iterations and MCMC repetitions for each run, and much longer burnin will be 
required for some data sets. Comparing a range of run durations may help to determine the appropriate 
run length, and it is always advisable to choose a longer burn-in and run length. To confirm that burn-in 
is adequate, it is also important check for convergence in values of summary statistics (particularly a, 
F, D, and the likelihood) that are estimated by the program, as recommended in the \textsc{structure} 
manual \citep{Pritchard:2007}. Additional independent replicate runs are of great importance as they 
limit the influence of stochasticity and increase the precision of the parameter estimates. That is 
especially true when using the Evanno method, which requires an estimate of variance. In at least one 
reanalysis we performed, only five replicate runs were used, which may explain the failure to reproduce 
results (the chosen \emph{K}) in this particular study. We recommend 20 replicates as used by \citet{Evanno:2005}.

\item[3] Proper reporting of the methods used to analyse \textsc{structure} results is vital for inferring \emph{K}. 
Whether the method outlined by \citet{Pritchard:2000} or by \citet{Evanno:2005} or both are used to select 
\emph{K} should be clearly stated, as well as any biological factors that have influenced the choice of \emph{K}. 
Special attention should be given to the comparison of \emph{K} = 1 versus greater values, as the Evanno method 
is not capable of performing this comparison. We advise that results are reported in the form of the graph of 
the natural logarithm of the likelihood of the data given \emph{K} (if the Pritchard method was used) and the 
$\Delta K$ graph (if the Evanno method was used) as well as the bar plot(s) showing individual assignments for the given 
\emph{K} or comparison across plausible \emph{K} values. Ideally, for full reproducibility of a study, membership 
coefficients for each individual should also be provided. These results should be examined within each replicate 
to determine how much stochasticity is present before runs are averaged, as well as after averaging all replicate runs.
\end{itemize}

\section{Conclusion}
A substantial proportion of \textsc{structure} results were not reproducible, despite the relative simplicity of 
the procedure, requiring only a genotype file and associated parameter settings. Our recommendations on how to 
archive data sets analysed with \textsc{structure} should reduce the component of nonreproducibility due to 
uncertainty of parameter choice or lack of clarity in the data analysed, but some discrepancies will no 
doubt still persist. We hope that scientists will increasingly acknowledge the concept of scientific 
reproducibility in the future and be aware of practices they can enact both for better data archiving 
and better implementation of other similar programs in their analyses.

%%% Local Variables:
%%% TeX-master: "thesis"
%%% TeX-PDF-mode: t
%%% End:
