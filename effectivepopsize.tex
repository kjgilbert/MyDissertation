\chapter{Evaluating methods for estimating local effective population size with and without migration}
\label{chap:effectivepopsize}

%%\section{Summary}

%%Effective population size is a fundamental parameter in population genetics, evolutionary biology, and conservation biology, yet its estimation can be fraught with difficulties. Several methods to estimate $N_e$ from genetic data have been developed that take advantage of various approaches for inferring Ne. The ability of these methods to accurately estimate $N_e$, however, has not been comprehensively examined. In this study, we employ seven of the most cited methods for estimating $N_e$ from genetic data (\textsc{colony2}, \textsc{cone}, \textsc{estim}, \textsc{mlne}, \textsc{onesamp}, \textsc{tmvp}, and \textsc{neestimator} including \textsc{ldne}) across simulated datasets with populations experiencing migration or no migration. The simulated population demographies are an isolated population with no immigration, an island model metapopulation with a sink population receiving immigrants, and an isolation by distance stepping stone model of populations. We find considerable variance in performance of these methods, both within and across demographic scenarios, with some methods performing very poorly. The most accurate estimates of $N_e$ can be obtained by using \textsc{ldne}, \textsc{mlne}, or \textsc{tmvp}; however each of these approaches is outperformed by another in a differing demographic scenario. Knowledge of the approximate demography of population as well as the availability of temporal data largely improves $N_e$ estimates.

\section{Introduction}
Effective population size ($N_e$) is defined by \citet{Wright:1931} as the size of 
an ideal population that has the same properties of genetic drift as the population at hand. 
$N_e$ is a vital parameter used throughout population genetics and evolutionary 
biology and applied widely in conservation genetics \citep{Schwartz:2007, Dudgeon:2012}. 
However, determining the true $N_e$ in a natural system can be a difficult task 
\citep{Serbezov:2012, Theunert:2012}. With the advent of affordable molecular tools 
in nonmodel systems, it has become increasingly popular to estimate $N_e$ using genotypic 
data \citep{Wang:2005, Palstra:2012}, and multiple approaches to do so have been developed. 
These methods function through a range of methodologies, which each entail one or several assumptions.

$N_e$ can be estimated by the fact that as a population decreases in effective size, 
it is increasingly subject to allele frequency changes due to random genetic drift and increased 
inbreeding \citep{Caballero:1994}. Quantifying these allele and genotype frequency changes can therefore 
be used to determine its $N_e$. Several population genetic properties can be used for 
this purpose: change in allele frequencies over time may reflect drift, and amounts of linkage 
disequilibrium (LD), homozygosity and heterozygosity, gene identity disequilibria, or coancestry 
may all deviate from expectation due to drift and inbreeding. Several methods use temporal data 
taken from two or more time points to estimate $N_e$ \citep{Nei:1981, Pollak:1983, Beaumont:2003, 
Wang:2003, Anderson:2005, Jorde:2007}. While allele frequency change over time gives the most 
direct estimate of drift, obtaining samples from multiple generations 
may be difficult in some systems. With genetic data from one point in time, $N_e$ can be 
estimated by inferring drift or inbreeding from measures such as LD or other non-allele frequency measures described above \citep{Hill:1981, Pudovkin:1996, Vitalis:2001a, Nomura:2008, Waples:2008, Wang:2009}, or even from a combination of some properties \citep{Tallmon:2008}. Details on the approach each program takes to estimate 
$N_e$ are fully described in the Methods.

Genetic drift is not the only process that may cause changes in these population genetic 
quantities. Migration is a major evolutionary force that can also affect changes in allele 
frequencies over time, amounts of LD, and other of the aforementioned properties. If changes 
due to migration are not accounted for by a method when estimating $N_e$, those 
changes may be attributed to drift and result in a biased and inaccurate estimate of effective 
size \citep{Wang:2003}. This bias from migration may be in either direction. If many 
foreign alleles are introduced into a population through migration, this will create the signature 
of a larger amount of drift than is occurring and therefore produce an underestimate of 
$N_e$. Migration in this case of higher population structure will also increase 
the amount of LD in the focal population. However, if populations are genetically similar 
(i.e., low \emph{F}$_{ST}$), then migration provides a larger number of parents than exist within the 
focal population, leading a small population to appear to be larger than it truly is and 
overestimating $N_e$. It is worth noting in this case, especially with conservation 
situations in mind, that the true $N_e$ is not represented by this larger size, since 
if migration were to be cut off, the focal population would again be subject to drift proportional 
to its isolated population size. Additionally, gene identity methods depend on gene flow for their 
calculations, and higher $N_e$\emph{m} values create the most accurate and precise %% Loren - not clear why higher Nem values create most accurate Ne values
$N_e$ estimates \citep{Vitalis:2001a, Vitalis:2001b}. Therefore the ability to detect 
whether drift alone or a combination of drift and migration is causing allele frequency 
changes in a population is the key to accurately estimating effective population size.

Method developers make every effort to clearly state the underlying assumptions for their 
approach and test its performance to ensure appropriate application. While most methods have 
been assessed concurrently with their publication, the test datasets used vary in design and 
are unique to each publication. Typically, these assessments use simulated, ideal circumstances, 
or violate only a small subset of assumptions for the method at hand (e.g., \citealt{Vitalis:2001c, 
Beaumont:2003, Wang:2003, Anderson:2005, Wang:2009, Do:2014}). Additionally, 
nonideal situations are generally not vetted across all possible estimation methods. In natural 
systems, which are frequently subject to migration, it can thus be difficult to know which method 
is best suited for the system in question to find the desired meaningful estimate of $N_e$. 
Several studies have therefore assessed the performance of a subset of estimators in specific 
empirical situations \citep{Barker:2011, Serbezov:2012, Holleley:2013, Macbeth:2013, 
Baalsrud:2014}. However, the true effective population size cannot be known in a natural 
system. Thus, evaluating the accuracy of a method is not possible through these comparisons.

Simulated datasets allow the true $N_e$ to be known, therefore allowing estimation 
accuracy to be assessed. By simulating more complicated demographic scenarios, non-ideal situations 
similar to those found in the real world can be addressed. Three such studies have attempted an 
evaluation of $N_e$ estimates in the presence of migration; however, they have used only 
a small subset of existing estimation methods in varying demographic scenarios \citep{Waples:2011, Neel:2013, Ryman:2013}. 
These studies give valuable insight into the performance of those 
methods when the assumption of a single isolated population is not true. They show that the performance 
of those methods can depend on assumptions about migration rates \citep{Waples:2011}, population 
substructure \citep{Ryman:2013}, and size of area sampled versus the typical distance within which 
mating occurs in populations under isolation by distance \citep{Neel:2013}. Despite the vital 
findings of these studies, the field still lacks a comprehensive comparison of existing estimation 
methods across identical, simulated datasets, which is the only way to make an informed decision on 
which method to apply for the most accurate answer.

Our study compares the seven most cited programs (which employ 14 estimation methods) for 
estimating effective population size. We simultaneously apply them over identical datasets 
designed specifically to cover a wide range of simulated real world migration scenarios in 
order to assess their performance. These programs, \textsc{colony2} \citep{Wang:2009}, \textsc{cone} \citep{Anderson:2005}, 
\textsc{estim} \citep{Vitalis:2001c}, \textsc{mlne} \citep{Wang:2003}, 
\textsc{neestimator} v2 \citep{Do:2014}, \textsc{onesamp} \citep{Tallmon:2008}, and 
\textsc{tmvp} \citep{Beaumont:2003}, are specifically designed with the purpose of estimating 
contemporary $N_e$. Our aim is to test each of the methods in an equivalent way 
to uncover the most accurate approach (or approaches) in estimating small scale, local 
$N_e$, as would be of interest in conservation studies. We only assessed programs 
that use genetic data and did not include additional programs that calculate $N_e$ 
as a long-term effective size on the scale of species or as a product of other parameters, 
such as $\theta$, since contemporary, local $N_e$ was our value of interest 
(e.g., Migrate \citep{Beerli:2001}, IM \citep{Hey:2004}, and Lamarc \citep{Kuhner:2006} among others). 
Our goal is to inform users as to which method(s) is best able to estimate $N_e$ in natural systems.

\section{Methods}
\subsection{Population Demographies}
Populations were simulated with the forward-time, individual-based simulation program 
\textsc{nemo} \citep{Guillaume:2006}. We modeled demographies with all populations 
at a constant size through time for isolation scenarios (no migration) and migration scenarios 
having either an island model metapopulation plus a nearby sink population, or a stepping stone 
case with linearly arrayed patches only sending migrants to the neighboring patch to represent 
an isolation by distance (IBD) scenario. Figure \ref{fig:ne-1} shows the design for the island model cases 
that have a metapopulation of 11 patches connected equally by migration (either $m = 0.01$ or $0.1$ throughout) 
and a sink population receiving unidirectional migration from a single patch in the metapopulation 
(either $m = 0.01$, $0.1$, or $0.25$). An isolated patch was also included in this simulation. The sampling 
and estimation of $N_e$ for these demographic cases (Fig. \ref{fig:ne-1} for the island model) is meant 
to represent situations one may encounter in a given real world system. In real applications, one 
may wish to estimate the $N_e$ of a population without realizing that it may be affected 
by migration. The local patch may be part of a large network of populations, or alternatively, the 
sampled population may itself be substructured. In some cases the presence of migration may be 
difficult to detect. We therefore assessed nine sampling situations across methods: isolation, 
sink (low, medium, and high migration from source), within metapopulation (a patch in the island 
model or a patch in the interior of the stepping stone model), an edge patch from the stepping 
stone model (one end of the linear array of patches), and across the whole metapopulation (sample 
across all patches randomly within the island model and the stepping stone model). Additionally, 
one estimation method, \textsc{mlne}, allows a migration source to be identified, for which we 
provided a range of source patches: the correct source population, a more distantly related source 
population, an entirely unrelated and isolated source population, the entire metapopulation combined 
as the migration source, or a case with no contemporary migration yet a historic migration source 
labeled incorrectly as a migration source (Fig. \ref{fig:ne-1}). This allowed us to further assess whether 
correctly or incorrectly identifying a migration source affected the performance of \textsc{mlne}.

\begin{figure}[]
\centering
\makebox[\textwidth]{
        \includegraphics[width=1.0\linewidth]{Figures/ne_fig1.pdf}}
\caption[~- Diagram showing the dispersal patterns for the simulated population demographies.]{Diagram showing the dispersal patterns for the island model migration simulations and the patches 
sampled for the different estimation cases. Low indicates $m = 0.01$, medium $m = 0.1$, and high $m = 0.25$.}
\label{fig:ne-1}
\end{figure}

In all cases we are interested in the local $N_e$ of a patch except when sampling 
the whole metapopulation, in which case the $N_e$ estimation would be expected to 
reflect the metapopulation $N_e$. Population sizes were set per patch and constant 
across patches with $N_e$ equal to census size by following the Wright-Fisher model 
(see below), and were the same across all patches within a simulation. For isolated (ideal) 
populations, we tested $N_e$ of 50, 500, and 5000 with 100 replicate simulations at 
$N_e = 50$ and 500 and 25 replicates at $N_e = 5000$. In the migration scenarios, 
only a local $N_e$ of 50 and 500 were tested, again over 100 replicate simulations each, 
as estimators performed poorly at $N_e = 5000$ in preliminary runs (due to computational 
requirements, some methods additionally were not assessed or assessed with fewer replicates at 
$N_e = 500$; see Results and Discussion). Several methods require that a prior be given 
for the estimate of $N_e$ for which we provided a value of true $N_e$ $\times$ 20 as 
the upper bound (see Table \ref{tab:ne-1}). Some methods we test estimate the inbreeding effective size 
($N_e$I) while others estimate the variance effective size ($N_e$V). However, 
our only source of nonrandom mating is due to the imposed migration, making these values 
equivalent \citep{Hill:1979}, and thus the estimates are comparable. Sample size was kept constant 
across all simulations at 250 individuals, which were obtained by sampling individuals in the 
desired generation who did not contribute to the next generation before regulation to carrying 
capacity. We additionally investigated a subset of analyses with a reduced sample size of 50 
individuals to see how this decreased performance of the methods that we found to be most accurate.

\begin{table}[]
\centering \footnotesize
\caption{Programs tested in this study with input parameters and other specifications used.}
\begin{tabular}{p{0.18\textwidth}|p{0.32\textwidth}|p{0.4\textwidth}}
\small{Program}           & \small{Citation}                                                                     & \small{Specifications} ($N_e = 50$ / 500 / 5,000) \\ \hline
\textsc{colony2} v2.0     & \citealt{Wang:2009}                                                                  & run length specified = 2                                                                                                                                       \\ \hline
\textsc{cone} v1.01       & \citealt{Anderson:2005}                                                              & \begin{tabular}[c]{@{}l@{}}10,000 iterations\\ prior Ne 2-1000 / 2-10,000 / 2-100,000\\ step 5 / 10 / 25\end{tabular}                                      \\ \hline
\textsc{estim} v1.2       & \citealt{Vitalis:2001a, Vitalis:2001b}                                               & prior max Ne 10,000$\dagger$ / 10,000 / 100,000                                                                                                                \\ \hline
\textsc{mlne} v1.0        & \citealt{Wang:2003}                                                                  & prior max Ne 1,000 / 10,000 / 38,250$\ddagger$                                                                                                                 \\ \hline
\textsc{neestimator} v2.0 & \begin{tabular}[c]{@{}l@{}}\citealt{Do:2014}\\ (\textsc{ldne} -- \citealt{Waples:2008})\end{tabular} & reporting results for MAF cutoff 0.05                                                                                                          \\ \hline
\textsc{onesamp}          & \citealt{Tallmon:2008}                                                               & \begin{tabular}[c]{@{}l@{}}50,000\\ iterations\\ prior Ne 2-1,000 / 2-10,000 (see text)\end{tabular}                                                           \\ \hline
\textsc{tmvp}             & \citealt{Beaumont:2003}                                                              & \begin{tabular}[c]{@{}l@{}}20,000-75,000\\ iterations\\ maxit 1,000-7,500;\\ thinning interval 5-10;\\ size distribution of updates 0.5; (see text)\end{tabular}
\end{tabular}
\caption*{$\dagger$ \footnotesize{The smallest maximum $N_e$ allowed by \textsc{estim} is 10,000.} \\
$\ddagger$ \footnotesize{\textsc{mlne} did not allow larger maximum $N_e$ due to memory limitations.}}
\label{tab:ne-1}
\end{table}

Individuals were simulated to represent genotypes with neutral microsatellite markers, as 
microsatellite data has been the main data type originally intended for programs used. The 
life cycles modeled in \textsc{nemo} were breeding combined with migration (defined by backward migration rates) followed by population 
regulation. In each generation, offspring are created from randomly chosen hermaphroditic parents 
until the population is of the desired size, \emph{N}. Generations were non-overlapping. We simulate 
a Wright-Fisher model where each offspring has an equal and random chance of coming from any given 
parent. By definition, this sets our $N_e$ equal to \emph{N}, though we clarify that 
this value differs slightly from the number of breeders that could be captured by viewing the population 
at a specific point in time \citep{Waples:2009}. Mutation rate was set at 10$^{-5}$ with a single 
step microsatellite mutation model, and 40 freely recombining polymorphic loci were used per individual, 
initialized at maximum variance randomly up to the maximum possible number of alleles of 256. Simulations 
were run until $F_{ST}$ was approximately 95\% of its equilibrium value (or mutation-drift balance reached 95\% 
equilibrium for isolation demographies), calculated from \citet{Whitlock:1992}.

\subsection{Temporal $N_e$ Estimation Methods}
We compared seven different programs that implement 14 different $N_e$ estimation 
methods (see Table \ref{tab:ne-1}). Seven of these methods employ temporal estimation using data from two 
or more points in time. For our datasets, we used two time points separated by one generation. 
Though it is known that increasing the number of generations between samples (up to at most eight 
generations between samples) improves estimation \citep{Wang:2003}, in a real sampling 
situation it is more difficult to obtain samples separated further in time. Therefore one generation 
between is most relevant to the current study, but is expected to produce conservative results 
in terms of performance of temporal estimators. We discuss the methods, and the defaults used, in turn.

\textsc{tmvp} uses a general method of genealogical inference with temporal genetic data to 
estimate recent changes in $N_e$ \citep{Beaumont:2003}. It incorporates both importance 
sampling methods and Markov chain Monte Carlo methods to sample independent genealogical 
histories from which a posterior distribution of $N_e$ is obtained with a Bayesian 
framework. Parameter settings varied across runs (Table \ref{tab:ne-1}) to ensure proper convergence of 
the analysis, where generally longer run times were required for higher migration cases. 
Posterior computations were performed using R scripts to calculate the Gelman-Rubin statistic 
to test for convergence, and the modes for the Bayes factor and the highest posterior density 
limits \citep{Barker:2011}. \textsc{mlne} \citep{Wang:2003} also uses temporal data, performs 
both a moment and maximum likelihood approach to estimate $N_e$, and simultaneously 
estimates migration rate along with $N_e$. Migration rates are estimated by providing 
an immigration source population at the previous temporal time point. \textsc{cone} \citep{Anderson:2005} 
is an importance sampling method that calculates the likelihood of $N_e$ with temporal 
data using Monte Carlo sampling under the coalescent model of \citet{Berthier:2002}, based on 
the coalescent of gene copies drawn from the more recent time sample.

\textsc{neestimator} v2.01 \citep{Do:2014} performs six different estimation methods, three 
of which use temporal data. These three approaches are based on calculations of variance of 
change in allele frequencies as defined by \citet{Nei:1981} and \citet{Pollak:1983}, which 
have slight variations in calculations and weightings of loci, and more recently by \citet{Jorde:2007} 
who account for bias in small sample sizes and low frequency alleles by 
weighting alleles differently.

\subsection{Single Sample $N_e$ Estimation Methods}
The seven remaining methodologies use samples from only one point in time. \textsc{estim} \citep{Vitalis:2001c} 
is a moment-based estimator that uses probabilities of one- and two-locus identity states. 
It also simultaneously estimates $N_e$ and migration. In small populations, drift 
can result in associations between occurring gene copies \citep{Hill:1968}, so \textsc{estim} 
defines a ``within-subpopulation identity disequilibrium`` to account for this correlation of 
gene identities \citep{Vitalis:2001a}. Runs were performed with random mating and 
mutation rate set to our known parameter of 10$^{-5}$. It is worth noting that the true mutation 
rate is unlikely to be known in a natural population, and we have not tested the sensitivity 
of the method to varying this parameter. \textsc{onesamp} is designed specifically for use 
with microsatellites and uses approximate Bayesian computation to estimate $N_e$ 
\citep{Tallmon:2008, Beaumont:2009}. It combines eight summary statistics specific 
to the inference of $N_e$, including several allele characteristics, heterozygosity, 
homozygosity, Wright's \emph{F}$_{IS}$, and a linkage characteristic, which are estimated for simulated 
datasets. Due to computational limitations from long run times, \textsc{onesamp} was tested 
on only a subset of $N_e = 500$ isolation replicates and not tested for $N_e = 5000$. 
\textsc{colony2} \citep{Wang:2009} implements a maximum likelihood method and estimates $N_e$ 
with either a heterozygote-excess method or a sibship relatedness estimate. We did not provide 
relatedness information with our data to this program, as it is not often available, and thus 
do not include \textsc{colony}'s sibship results in our methods comparisons.

The remaining three single sample estimators from \textsc{neestimator} use LD information, as previously 
implemented in the program \textsc{ldne} \citep{Waples:2008}, another form of the heterozygote-excess 
method as described in \citet{Pudovkin:1996}, and a method by \citet{Nomura:2008} using a parent-based coancestry 
estimate among individuals in one cohort. \textsc{ldne} estimates $N_e$ based on the 
amount of LD within a deme \citep{Hill:1981} and corrects for bias in calculating over a wide range of 
sample sizes \citep{Waples:2008}. Confidence intervals were obtained from the jackknife option, 
as parametric CIs are believed to be too narrow when locus pairs are not entirely independent. 
\textsc{neestimator} allows minimum allele frequency (MAF) cutoff values to be chosen for the analysis 
for which we chose a cutoff value of 0.05 as recommended by the authors to ensure that results were 
not being driven by the presence of rare alleles in the data.

\subsection{Composite Estimates}
Because each method uses different information from the data, we also tested if a combination of 
the most informative methods improved the accuracy and precision of estimates. We combined
estimates of $N_e$ across a subset of the methods found to perform best to additionally 
create a more fair comparison between single sample methods that would otherwise use only half as 
much data compared to temporal methods. We averaged the estimates of $\frac{1}{2 N_e}$ to produce 
the composite estimates. The composite estimates tested averaged the results from combinations of 
(1) \textsc{mlne}'s likelihood method, (2) \textsc{mlne}'s moment method, (3) \textsc{ldne}'s 
estimate calculated at each of the two time points used for the temporal methods, and (4) \textsc{tmvp}.

\subsection{Analysis of Estimates}
Estimates from each method were converted to values of 1/(2$N_e$), as this is the rate at 
which genetic drift occurs and therefore the value of interest against which we measure error 
(see Wang and Whitlock 2003). We assessed methods by calculating root mean square error (RMSE) 
of this value. RMSE accounts for both bias and precision of the methods, estimating the typical 
error in an estimate:
\begin{equation}
RMSE = \sqrt{ \frac{1}{n} \displaystyle\sum_{i=1}^{n} \Big( \frac{1}{\hat{N_e}} - \frac{1}{True N_e} \Big)^2}
\end{equation}
where $\hat{N_e}$ equals the estimated $N_e$ of the population and \emph{n} the
number of replicate estimates. Because we simulated a Wright-Fisher model, true $N_e$ 
is known in our cases, an advantage of simulated data, allowing an accurate and precise measure of
the error in any given method. Whole metapopulation $N_e$ was
calculated according to \citet{Wright:1943} where 
$N_{e ~ metapopulation} = \frac{N_{e ~ patch} ~ \times ~ number ~ patches}{1 - F_{ST}}$
, and \emph{F}$_{ST}$ is calculated according to \citet{Wright:1931}: 
$F_{ST} = \frac{1}{4N_em + 1}$.

The amount of drift that even an ideal population experiences in a given generation may 
differ from the expectation of the Wright-Fisher model, because of random deviations from 
the expected distribution of reproductive success \citep{Waples:2009}. This generates 
an additional source of error in the estimation of the $N_e$ of the system using 
data from a single generation. However, the goal of most studies is to estimate the $N_e$ 
(or $\frac{1}{N_e}$) of a population over all instances of the drift process; this expected 
$N_e$ is what helps best predict the amount of drift in a population in future generations 
(for conservation purposes) or for comparison to other species with different mating systems or 
life histories (for basic science reasons). Therefore we use the expected $N_e$ 
(i.e., \emph{N} from a Fisher-Wright population) as our standard of comparison when calculating 1/(2$N_e$ ).

RMSE was averaged across all scenarios and then within groupings of scenarios 
(isolation, island model, stepping stone) to describe methods that performed best overall. 
We calculated RMSE after removing estimates where the program indicated a failed run, as 
described in the reference for each method, or estimates that indicated invalid answers. (\textsc{estim} and the
temporal \textsc{neestimator} methods produced zero estimates in a subset of cases; 
personal communication with R. Vitalis and C. Do confirmed that these are cases where the 
user is best to apply a different estimation method.) Additionally, as described by \citet{Do:2014}, 
negative estimates from \textsc{neestimator} were taken to indicate an infinitely large $N_e$. 
Confidence intervals for RMSE were obtained by bootstrapping across $N_e$ estimates within a 
demographic scenario with 10,000 replicate bootstrap samples each. Methods were further assessed per 
specific demographic scenario individually. We also examined the coverage probability of containing 
the true value of $N_e$ for confidence intervals produced from each estimate.

\section{Results}
Results for all programs and scenarios are reported throughout the figures and tables in the text 
and the supplemental material. Figure \ref{fig:ne-2} shows the RMSE of all methods for the isolation (no migration) 
cases, the island model migration scenarios, and the stepping stone scenarios respectively. Numeric 
values of RMSE for all methods and scenarios are provided in Tables \ref{tab:ne1}--\ref{tab:ne5} while Table \ref{tab:ne-2} provides 
averaged RMSEs for the best-performing methods and Table \ref{tab:ne-3} for their composite estimates. 
Coverage probabilities for confidence intervals are shown in Tables \ref{tab:ne6}--\ref{tab:ne8}, and the proportion 
of times that each estimator produced an infinite estimate of $N_e$ is shown in Figure \ref{fig:supp_propinf}. 
Point estimates of $N_e$ with 95\% confidence intervals for all replicates across all 
methods and scenarios are visualized to show accuracy and precision in Figures \ref{fig:supp_ldne}--\ref{fig:supp_avg5}.

\begin{figure}[]
\centering
\makebox[\textwidth]{
        \includegraphics[width=0.9\linewidth]{Figures/ne_fig2.pdf}}
\caption[~- RMSE for all methods across all scenarios.]{RMSE for all methods across (A) isolation scenarios, (B) island-model migration scenarios, and (C) 
IBD stepping-stone scenarios. The only sink case shown for island model cases is $m = 0.1$ immigration rate from 
the metapopulation, as all sink cases showed similar results. The edge patch sampling case is not shown for 
the stepping stone model, as results were very similar to the within case. Note that the scale on the y-axis 
in panel (A) differs from those in panels (B) and (C). Bootstrapped 95\% confidence intervals are shown only 
in (A), as the intervals are smaller or equal to the point size for all but ten estimates of RMSE in (B) and (C).}
\label{fig:ne-2}
\end{figure}

The best (i.e., lowest) RMSE on average across all possible demographic scenarios was \textsc{mlne}'s 
likelihood method (RMSE = 0.00240; see Table \ref{tab:ne-2}). However, because RMSE varied greatly across demographic 
scenarios, we describe results per class of scenario in the following sections. We group results 
into demographic classes of no migration, which contains scenarios of isolated populations, and 
migration scenarios which contain groupings of single patches as either sinks or within the island 
model metapopulation, as well as the whole metapopulation taken as one population, and the stepping 
stone models again with either single patches within the linear stepping stone array, or the whole 
metapopulation as one population. \textsc{ldne} had the lowest average RMSE in cases with no migration 
(Table \ref{tab:ne-2}). \textsc{mlne}'s various estimation methods did best in the island model and stepping stone 
migration scenarios, with one demographic case where \textsc{tmvp} performs comparably 
(whole metapopulation case with local $N_e = 50$).

\begin{table}[]
\centering \tiny
\caption{RMSE averaged over groups of demographic scenarios for the best performing estimation 
methods.}
\begin{tabular}{llllllllllll}
                               & \begin{tabular}[c]{@{}l@{}}Overall \\ Average \\ RMSE \end{tabular}           & \begin{tabular}[c]{@{}l@{}}Isolated \\ N50\end{tabular} & \begin{tabular}[c]{@{}l@{}}Isolated \\ N500\end{tabular} & \begin{tabular}[c]{@{}l@{}}One Patch \\ Migration \\ N50\end{tabular} & \begin{tabular}[c]{@{}l@{}}One Patch \\ Migration \\ N500\end{tabular} & \begin{tabular}[c]{@{}l@{}}Metapop'n \\ Migration \\ N50\end{tabular} & \begin{tabular}[c]{@{}l@{}}Metapop'n \\ Migration \\ N500\end{tabular} & \begin{tabular}[c]{@{}l@{}}One Patch \\ IBD N50\end{tabular} & \begin{tabular}[c]{@{}l@{}}One Patch \\ IBD N500\end{tabular} & \begin{tabular}[c]{@{}l@{}}Metapop'n \\ IBD N50\end{tabular} & \begin{tabular}[c]{@{}l@{}}Metapop'n \\ IBD N500\end{tabular} \\ \hline \hline
\textsc{ldne}                  & 0.00828                         & \cellcolor[HTML]{C0C0C0}0.00163 & \cellcolor[HTML]{C0C0C0}0.00094 & 0.00146                         & 0.00409                         & 0.00074                         & 0.00009                         & 0.00263                         & 0.01977                         & 0.04595                         & 0.00841                         \\ \hline
\begin{tabular}[c]{@{}l@{}}\textsc{mlne} \\ likelihood\end{tabular}       & \cellcolor[HTML]{C0C0C0}0.00240 & 0.00341                         & 0.00141                         & 0.00347                         & 0.00053                         & \cellcolor[HTML]{C0C0C0}0.00021 & \cellcolor[HTML]{C0C0C0}0.00004 & 0.00351                         & \cellcolor[HTML]{C0C0C0}0.00606 & \cellcolor[HTML]{C0C0C0}0.00011 & 0.00048      \\ \hline
\begin{tabular}[c]{@{}l@{}}\textsc{mlne} lik. \\ with source\end{tabular} & 0.00242                         & --                              & --                              & \cellcolor[HTML]{C0C0C0}0.00119 & 0.00054                         & --                              & --                              & \cellcolor[HTML]{C0C0C0}0.00115 & 0.00785                         & --                              & --                              \\ \hline
\begin{tabular}[c]{@{}l@{}}\textsc{mlne} \\ moment\end{tabular}           & 0.00247                         & 0.00349                         & 0.00202                         & 0.00229                         & \cellcolor[HTML]{C0C0C0}0.00020 & 0.00041                         & 0.00006                         & 0.00214                         & 0.00952                         & 0.00031                         & \cellcolor[HTML]{C0C0C0}0.00036 \\ \hline
\begin{tabular}[c]{@{}l@{}}\textsc{mlne} mom. \\with source\end{tabular} & 0.00299                         & --                              & --                              & 0.00203                         & 0.00054                         & --                              & --                              & 0.00181                         & 0.00870                         & --                              & --                              \\ \hline
\textsc{tmvp}                  & 0.00281                         & 0.00349                         & 0.00176                         & 0.00236                         & 0.00034$\dagger$                & \cellcolor[HTML]{C0C0C0}0.00020 & 0.00059                         & 0.00205                         & 0.01129                         & 0.00037                         & --$\dagger$       \\ \hline             
\end{tabular}
\caption*{\small{Gray boxes indicate the method with the lowest RMSE for that demography grouping. 
Dashes indicate cases where $N_e$ was not estimated. See text for descriptions of 
demographic groupings.} \\
\footnotesize{$\dagger$ Parameter settings did not converge for \textsc{TMVP} runs, see text.}}
\label{tab:ne-2}
\end{table}

\subsection{No Migration Scenarios}
The three isolation scenarios ($N_e = 50$, $500$, $5000$), in which populations experience 
no migration, showed \textsc{ldne} to consistently have the lowest RMSE. At $N_e = 500$, \textsc{onesamp} closely compares to \textsc{ldne}, but is one of the worst methods at $N_e = 50$. \textsc{estim} has a low RMSE 
at both $N_e = 500$ and $5000$, but at $N_e = 50$ \textsc{estim} largely overestimates 
$N_e$ with many very large or infinite estimates. \textsc{colony}'s heterozygote 
excess method tends to estimate infinite $N_e$ in all cases, reflected by its 
large RMSE at $50$ and $500$ (Fig. \ref{fig:ne-2}A). \textsc{mlne}'s likelihood estimate performs next best 
after \textsc{ldne} at $500$ and $5000$ while the remaining majority of methods performed equally 
well at size $50$. It is worth noting that the heterozygote excess method \citep{Pudovkin:1996} 
and the coancestry method \citep{Nomura:2008} as implemented in \textsc{neestimator} perform poorly at all population sizes.

For the most accurate methods in the isolation scenarios, \textsc{ldne} and \textsc{mlne} 
likelihood, it is useful to examine the coverage probability of confidence intervals around 
each estimate. Confidence intervals from \textsc{ldne} captured the true $N_e$ 80\% 
of the time for $N_e = 50$ and 68\% for $N_e = 500$, though six percent of 
those intervals spanned to infinity at their upper end, and 24\% for $N_e = 5000$ with 
24\% of those upper confidence intervals spanning to infinity. \textsc{mlne} likelihood confidence 
intervals contained the true value of $N_e$ 82\%, 89\%, and 16\% of the time, respectively, 
however at $N_e = 500$, the upper bound on the confidence interval spanned to the maximum 
prior $N_e$ provided 39\% of the time. Results for confidence intervals of all methods 
for no migration demographic cases are shown in Table \ref{tab:ne6}.

\subsection{Migration Scenarios}
RMSEs for the island model and stepping stone migration cases showed higher variation than did 
the isolation cases (Fig. \ref{fig:ne-2}). Note that the y-axes on Figure \ref{fig:ne-2}B and C have a greater range than 
in Figure \ref{fig:ne-2}A and remain on a log scale. \textsc{neestimator}'s coancestry and heterozygote excess 
methods again fall out as some of the poorer performing methods. In the scenarios where the whole 
metapopulation has been sampled randomly (ignoring existing population substructure), many methods 
show drastic change in their performance, either estimating $N_e$ much more poorly (e.g., 
\textsc{neestimator}'s methods according to \citet{Pollak:1983} and \citet{Nei:1981}), varying from 
poorer to better depending on migration rate (\textsc{estim}), or improving (\textsc{mlne} moment and at a 
smaller $N_e$, \textsc{tmvp}). Several of these changes in performance are a result of 
infinite estimates being produced.

Among the migration scenarios, we averaged RMSE across single patch cases (a sink population in 
the island model scenarios or an edge population in the stepping stone scenarios grouped respectively 
with the population sampled within the metapopulation) and for the metapopulation as a whole at local 
patch size of 50 or 500 across the various migration rates (Table \ref{tab:ne-2}). No method was consistently 
best across these scenarios. \textsc{mlne}'s likelihood method with a known migration source performed 
best on single patch cases at $N_e = 50$ for both the island model and the
stepping stone model scenarios, and \textsc{mlne}'s likelihood method with no source identified 
showed the best RMSE for the whole metapopulation at $N_e = 500$ in the island model 
scenarios, but includes many estimates at the upper bound of the prior, effectively leaving 
\textsc{mlne}'s moment method as the more accurate method in this case. In the stepping stone scenarios, 
again \textsc{mlne}'s likelihood method is best in single patch cases at both $N_e = 50$ and 500 
while the moment method is best for the metapopulation cases, once accounting for $N_e$ estimates 
at the upper bound of the prior.

Despite not having the lowest RMSE for migration cases on average, \textsc{ldne} still produces 
accurate $N_e$ estimates and has a lower RMSE in individual patch sampling scenarios, particularly 
at lower migration rates. Its higher average RMSE is driven by cases where the whole metapopulation is 
sampled or there is a combination of high migration rate and larger population size ($m = 0.1$, $N_e = 500$) 
in which cases \textsc{ldne} produces infinite or inaccurate estimates of $N_e$. In these demographic 
scenarios, \textsc{mlne} or \textsc{tmvp} are needed to give appropriate estimates.

Additionally, when examining the coverage probability of confidence intervals containing the true 
$N_e$ from these methods, \textsc{ldne} shows greater coverage than \textsc{mlne} for 18 of the 20 
island model sampling scenarios and 16 of the 18 stepping stone scenarios. However, \textsc{ldne} also 
produces confidence intervals spanning to infinity in eight of those sampling scenarios for the island 
model and seven for the stepping stone model. Coverage probabilities for all methods are shown in 
Tables \ref{tab:ne6}-\ref{tab:ne8} for all methods and demographic scenarios.

\subsection{Composite Methods and Sample Size Comparisons}
The composite estimates of $N_e$ showed only slight improvements in RMSE in a subset of 
methods and demographic scenarios (Table \ref{tab:ne-3}). Averaged over all scenarios, the composite estimate from 
\textsc{mlne}'s likelihood and moment estimates showed the lowest RMSE. Within grouped demographic 
scenarios, only in the single patch cases for \emph{N}$_e = 50$, the IBD stepping stone metapopulation 
at $N_e = 500$, and both isolation cases did a composite estimate improve over individual methods' 
performances. As expected, the composite estimate of \textsc{ldne} from two time points always improved upon 
\textsc{ldne} at one time point for all demographic scenarios, though did not change its rank performance 
relative to the best-performing methods averaged overall or for the migration cases (\textsc{ldne} was 
already ranked best in isolation scenarios). However, no other composites showed consistent improvement 
across demographies. The composites are generally poorer than individual methods for the whole 
metapopulation sampling cases or at larger population sizes ($N_e > 500$).

Additionally, comparing RMSEs for our results to those from estimates using a sample size of only 50 
individuals showed the expected result of a decrease in precision and accuracy in both isolation and 
migration scenarios (see Fig. \ref{fig:supp_rmsesamp}).


\begin{table}[]
\centering \tiny
\caption{RMSE averaged over groups of demographic scenarios for composite estimates from the 
best performing methods.}
\begin{tabular}{llllllllllll}
                               & \begin{tabular}[c]{@{}l@{}}Overall \\ Average \\ RMSE \end{tabular}           & \begin{tabular}[c]{@{}l@{}}Isolated \\ N50\end{tabular} & \begin{tabular}[c]{@{}l@{}}Isolated \\ N500\end{tabular} & \begin{tabular}[c]{@{}l@{}}One Patch \\ Migration \\ N50\end{tabular} & \begin{tabular}[c]{@{}l@{}}One Patch \\ Migration \\ N500\end{tabular} & \begin{tabular}[c]{@{}l@{}}Metapop'n \\ Migration \\ N50\end{tabular} & \begin{tabular}[c]{@{}l@{}}Metapop'n \\ Migration \\ N500\end{tabular} & \begin{tabular}[c]{@{}l@{}}One Patch \\ IBD N50\end{tabular} & \begin{tabular}[c]{@{}l@{}}One Patch \\ IBD N500\end{tabular} & \begin{tabular}[c]{@{}l@{}}Metapop'n \\ IBD N50\end{tabular} & \begin{tabular}[c]{@{}l@{}}Metapop'n \\ IBD N500\end{tabular} \\ \hline \hline
\textsc{ldne / ldne}                                                                                              & 0.00774                         & \cellcolor[HTML]{9B9B9B}0.00120 & \cellcolor[HTML]{9B9B9B}0.00090 & 0.00128                         & 0.00276                         & 0.00074                         & 0.00009                         & 0.00235                         & 0.01939                         & 0.04407                         & 0.00858                         \\ \hline
\begin{tabular}[c]{@{}l@{}}\textsc{ldne /ldne} \\ / \textsc{tmvp}\end{tabular}                                    & 0.00568                         & 0.00141                         & 0.00109                         & 0.00135                         & 0.00104                         & 0.00056                         & 0.00014                         & 0.00183                         & 0.01698                         & 0.02826                         & --$\dagger$                     \\ \hline
\begin{tabular}[c]{@{}l@{}}\textsc{mlne} lik. \\ / moment\end{tabular}                                            & \cellcolor[HTML]{9B9B9B}0.00197 & 0.00326                         & 0.00170                         & 0.00159                         & 0.00032                         & 0.00031                         & \cellcolor[HTML]{C0C0C0}0.00005 & 0.00150                         & \cellcolor[HTML]{C0C0C0}0.00778 & 0.00019                         & \cellcolor[HTML]{9B9B9B}0.00005 \\ \hline
\begin{tabular}[c]{@{}l@{}}\textsc{mlne} lik. \\ / moment \\ \textsc{tmvp} \end{tabular}                          & 0.00210                         & 0.00330                         & 0.00172                         & 0.00149                         & \cellcolor[HTML]{C0C0C0}0.00030 & \cellcolor[HTML]{C0C0C0}0.00027 & 0.00018                         & 0.00130                         & 0.00888                         & \cellcolor[HTML]{C0C0C0}0.00013 & --$\dagger$                     \\ \hline
\begin{tabular}[c]{@{}l@{}}\textsc{mlne} lik. \\ / moment \\ \textsc{ldne / ldne}\end{tabular}                    & 0.00455                         & 0.00171                         & 0.00120                         & \cellcolor[HTML]{9B9B9B}0.00083 & 0.00140                         & 0.00052                         & 0.00006                         & 0.00121                         & 0.01342                         & 0.02200                         & 0.00434                         \\ \hline
\begin{tabular}[c]{@{}l@{}}\textsc{mlne} lik. \\ / moment \\ \textsc{ldne / ldne} \\ / \textsc{tmvp}\end{tabular} & 0.00396                         & 0.00203                         & 0.00130                         & 0.00099                         & 0.00067                         & 0.00045                         & 0.00008                         & \cellcolor[HTML]{9B9B9B}0.00112 & 0.01316                         & 0.01692                         & --$\dagger$            \\ \hline        
\end{tabular}
\caption*{\small{Gray boxes indicate the method with the lowest RMSE for that demography 
grouping. Darker gray boxes indicate cases where the composite estimate showed a lower RMSE 
than an individual method's estimate. Dashes indicate cases where $N_e$ was not estimated. See 
text for descriptions of demographic groupings.} \\
\footnotesize{$\dagger$ Parameter settings did not converge for \textsc{TMVP} runs, see text.}}
\label{tab:ne-3}
\end{table}

\subsection{Estimating $N_e$ and Migration Rates}
We found no consistent effect of different source populations on the performance of \textsc{mlne} when 
estimating $N_e$ (see Fig. \ref{fig:supp_rmsemig}). RMSE for \textsc{mlne}'s likelihood method at $N_e = 50$ 
was improved in cases where either a correct or incorrect source population was identified. At a local size of 
500, these results changed, where improvement only occurred in certain cases for the island model 
demography, or not at all for the stepping stone cases. \textsc{mlne}'s moment method showed different 
but equally inconsistent results. For $N_e = 50$, different demographies produced $N_e$ 
estimates that were more or less accurate with a migration source. While at population size 500, not providing 
any migration source was consistently better than providing one for the island model cases, the 
opposite was true in the stepping stone model where providing a correct or closely related source both 
consistently improved upon $N_e$ estimates (where the correct source provided was consistently 
better than the related source).

Finally, our last assessment pertained to the accuracy of migration rates that \textsc{mlne} estimates 
alongside its estimates of $N_e$. While the method showed high performance in its estimates of 
$N_e$, the migration rates it estimated in our cases were less accurate. When the correct migrant 
source was provided, migration rate estimates were most accurate at lower population sizes ($N_e = 50$) 
and lower migration rates ($m = 0.01$ -- $0.1$). This accuracy decreased with increasing population size 
and increasing migration rates (Fig. \ref{fig:supp_compmig}). When migration was cut off from the source to the focal patch 
prior to the generations over which sampling occurred, but still (incorrectly) identified as a source for 
estimating $N_e$, migration rate estimates were even less accurate. The historic migration rate 
(prior to migration being cut off) was more closely estimated than the true migration rate over the 
sampling period ($m = 0$). $N_e$ estimates however, were either not affected or only slightly 
reduced in accuracy in these scenarios.

\section{Discussion}
Effective population size is clearly useful in population genetics and conservation biology, yet as 
we have shown, its estimation is fraught with difficulties. By definition, estimating effective 
population size derives from the ability to quantify the amount of random genetic drift a population 
experiences. Migration complicates this quantification of drift by changing allele frequencies in 
less predictable ways. Many methods have been developed to estimate $N_e$ in natural 
systems, and our results show that these methods vary highly in performance across demographic scenarios 
of differing migration rates and population sizes. Some methods show consistently poor performance and 
are best avoided, while we can recommend several methods that show the most accuracy and precision in 
our demographic scenarios (summarized in Table \ref{tab:ne-4}). Because $N_e$ is used in many population 
genetic equations as well as to inform conservation decisions (e.g., \citealt{Shaffer:1981, Rieman:2001}), it 
is vital that we estimate it as accurately as possible.

Several methods were consistently poor estimators across all of our demographic scenarios, that is 
there was always a better method to choose from. The poorest performing methods are reassuringly 
mainly those with the fewest number of citations out of the methods we evaluated. \textsc{neestimator}'s 
coancestry and heterozygote excess methods were two such methods that stand out as consistently poor 
performers. However, the third most highly cited program, \textsc{onesamp}, has been shown to perform 
suboptimally in our demographic scenarios. We found no cases where it showed the lowest RMSE of all 
possible methods in any of our scenarios. Given that this method uses a Bayesian approach, adjusting 
the prior expectation may improve results, as is claimed by \citet{Holleley:2013}. However, it seems 
unlikely that the majority of systems in which $N_e$ is estimated are studied well enough to 
create an informative prior; therefore we believe the wide prior we provided is a valid assessment of 
this method. Additionally, this program is subject to very long-run times, a limitation in our study 
that prevented assessing the method at all of our larger $N_e$ cases since individual run 
times were greater than 30 days at these sizes.

Heterozygote excess approaches understandably performed poorly at higher $N_e$ values, as 
the theory upon which this estimation is based will only detect an excess of heterozygotes when 
a population is effectively small. Therefore, at larger sizes, a lack of heterozygote excess may 
not allow a large $N_e$ to be distinguished from an even larger $N_e$, explaining 
the lack of accuracy in those cases. Yet even when $N_e = 50$, these methods (heterozygote 
excess methods of \textsc{colony2} and \textsc{neestimator}) were worse than others. Though 
\textsc{colony2}'s likelihood approach, which is implemented alongside its heterozygote excess method, 
largely produced infinite estimates of $N_e$ with no sibship data provided, this method is 
expected to improve when such data are available.

The methods which we found to perform best were \textsc{ldne}, \textsc{mlne}, and \textsc{tmvp}. 
These methods were shown to be more reliable in terms of most consistently estimating $N_e$ 
across demographic scenarios, most accurate, and least biased. Fortunately, this fits with usage 
patterns; \textsc{ldne} is the most highly cited method, and \textsc{mlne} the second most cited, 
according to Web of Science as of January 12, 2015. \textsc{ldne} performs best in situations with 
little to no migration ($m = 0.01$). This has been previously described by Waples and England (2011) 
where they found \textsc{ldne} to perform adequately up to migration rates of approximately 5--10\%. 
With migration rates of 1\%, we find its estimates of $N_e$ are still accurate and have 
appropriate confidence intervals at lower population sizes. Interestingly, \textsc{ldne}'s performance 
deteriorates greatly at higher migration rates when the population size is larger ($N_e = 500$), 
in which case it would not be advisable to apply \textsc{ldne}.

%% Ne estimation methods citation table
\begin{table}[]
\centering \footnotesize
\caption{Summarized results across methods.}
\begin{tabular}{p{0.16\textwidth}|p{0.3\textwidth}|p{0.5\textwidth}}
\small{Program}			& \small{Recommendation}		        & \small{Reasons} \\ \hline
\textsc{colony2} v2.0 	&	Heterozygote excess method: not recommended	& Overestimates or estimates infinite values \\
						& 	Sibship method: not tested	        & --- \\ \hline
\textsc{cone} v1.01 	&	Not recommended for migration scenarios	& Works well in ideal scenarios, often fails in migration scenarios, particularly with large $N_e$ \\ \hline
\textsc{estim} v1.2 	&	Not recommended	                    & Overestimates in ideal and migration scenarios and underestimates in metapopulation and large $N_e$ stepping stone scenarios \\ \hline
\textsc{mlne} v1.0 		&	Recommended	                        & Moment method superior in migration scenarios and moment and likelihood methods both perform well over all scenarios, though estimates of migration rate often inaccurate \\ \hline
\textsc{neestimator} v2.0 	&	\textsc{ldne}: recommended      & Superior performance in ideal scenarios, performs well in migration scenarios with low m and smaller $N_e$ \\
						& 	Other estimators: not recommended	& Inconsistent or inaccurate estimates \\ \hline
\textsc{onesamp} 		&	Not recommended	                    & Overestimates small Ne, long run times at large $N_e$ prohibited testing \\ \hline 
\textsc{tmvp}			&	Recommended with reservation	    & Works well at small $N_e$ in ideal and migration scenarios as well as for whole metapopulation, but long run times and difficult to establish working parameters sets for analyses \\
\end{tabular}
\label{tab:ne-4}
\end{table}

\citet{Neel:2013} evaluated \textsc{ldne} in the context of continuously distributed populations, 
examining the effect of sampling too many or too few individuals relative to the area over which 
individuals mate randomly. They found that as long as sampling was restricted to the scale of local 
breeding (i.e., sampling window = breeding window), \textsc{ldne} performed well. However, with an 
increasing sampling window, $N_e$ was underestimated. We see this underestimation occur in 
our stepping stone model as well, when we sample across the whole metapopulation. But this 
underestimation is also seen even when our sampling window matches the scope of local breeding 
at larger population sizes ($N_e = 500$). Similarly, in the island model demography at 
a metapopulation migration rate of $m = 0.1$ when $N_e = 500$, $N_e$ is also 
underestimated, as well as only to a slight extent in the stepping stone model at $m = 0.25$ and $N_e = 50$.

Cases with population substructure are of particular significance. In our cases of sampling across 
the metapopulation in the island model, \textsc{ldne} produces infinite estimates of $N_e$ at 
both migration rates tested, while in the stepping stone model it either underestimates $N_e$ 
(when $m = 0.01$) or again produces infinite estimates. Even at some higher migration cases, \textsc{ldne} 
is still not the worst method, and it often underestimates the true $N_e$, which in terms of 
conservation efforts only results in a conservative answer. Only when the method fails to produce a 
finite $N_e$ estimate in the substructure cases may this become dangerous, as one cannot 
differentiate whether the population is truly large enough not to experience detectable drift or if 
the method is unable to estimate an accurate $N_e$.

\textsc{mlne}, which simultaneously implements a moment and a likelihood estimate of $N_e$, 
appears to be the best program in cases of higher migration and population substructure. One might 
expect higher migration rates to reflect a subpopulation's $N_e$ being estimated closer to 
that of the whole metapopulation as m increases, however this trend is not generally seen in either 
the island model or stepping stone model scenarios. Most methods underestimate the local $N_e$, 
except for \textsc{mlne}'s likelihood method that does show an average increase in local $N_e$ 
with increasing migration in the stepping stone model. \citet{Ryman:2013} evaluated a temporal method 
in cases of migration and population substructure and found bias towards underestimating $N_e$. 
\textsc{mlne}'s likelihood method shows this trend only in the stepping stone scenarios at $N_e = 500$ 
when using a single focal patch (i.e., not the entire metapopulation). The moment method of \textsc{mlne} shows 
a similar bias toward underestimating $N_e$ in those same stepping stone single population 
cases at both $N_e = 50$ and 500 as well as very slight underestimation of the true $N_e$ 
in the island model sink populations with higher migration rates. Barring those cases, \textsc{mlne} shows 
very accurate estimates of $N_e$. As described by \citet{Ryman:2013}, estimating the whole 
metapopulation $N_e$ may be improved by increasing sampling. Without increasing our sample size, 
we still find \textsc{mlne}'s moment method to estimate the whole metapopulation $N_e$ quite 
accurately, or when less accurate, is still the most accurate estimate out of all the methods.

\textsc{tmvp} produced very accurate $N_e$ estimates in the migration cases. Despite 
its adequate estimating ability, \textsc{tmvp} proved to be one of the most difficult analyses to conduct. 
Because the method uses importance sampling and MCMC, the parameter settings for each run must be set to 
allow for the results to converge on one estimate of $N_e$, which proved incredibly difficult at 
some of our larger $N_e$ and m cases. For those that we did succeed in finding a suitable set of 
run parameters, run times exceeded several weeks, therefore we did not run \textsc{tmvp} for all 
replicates of simulated datasets.

For higher migration, higher $N_e$, and population substructure cases, our results 
lend credit to applying \textsc{mlne} to obtain the most accurate $N_e$ estimates. However, 
it is important to point out that the coverage probability of \textsc{mlne}'s confidence intervals 
for the likelihood method (the moment method produces no confidence
intervals) were inaccurate. \textsc{ldne}'s coverage probability was better than \textsc{mlne}'s 
on average. Thus, even though \textsc{mlne} provides accurate $N_e$ estimates, users are 
cautioned against trusting the confidence intervals to reliably include the true value of $N_e$. 
This has been described previously both for \textsc{mlne} and \textsc{tmvp} \citep{Tallmon:2004}, and 
though we did find poor coverage for \textsc{tmvp}, it was less than the overconfidence of 
\textsc{mlne}'s confidence intervals.

An additional promised feature of \textsc{mlne} is to estimate the migration rate from an 
identified source population which it also makes use of to improve its estimate of $N_e$ 
when migration is occurring. When we provided a source population, however, there was only noticeable 
improvement in estimation of $N_e$ for those cases having a smaller $N_e$. 
Interestingly, incorrectly identifying the source of immigration did not reduce the performance of 
\textsc{mlne}, as might have been expected. Therefore, lack of an identified source of migration 
should not necessarily deter users from applying \textsc{mlne}.

If it is suspected that migration rates may have changed in the recent past, we found that 
\textsc{mlne} may produce less accurate $N_e$ estimates. Moreover, relying on \textsc{mlne} 
to estimate accurate migration rates is not advised. Estimates of migration rate may improve with 
more generations between sampling points, but this topic requires further study.

Composite estimates did not substantially improve the ability to estimate $N_e$. 
Because each of these approaches uses different information from the same data, we had predicted 
that combining information could improve accuracy in $N_e$ estimates. Though slight 
improvements in RMSE were seen, these results cannot be extrapolated widely beyond our migration 
scenarios. Poorly performing methods erred more due to bias, rather than imprecision. Averages 
of biased results can only increase precision around an incorrect value, and any biases in 
opposite directions across methods that might effectively cancel out to give accurate results 
are likely to be idiosyncratic. It is clear that any unbiased single sample estimator would 
improve by averaging it over two temporal samples, as our results for \textsc{ldne} show. 
However, one of the major highlights of \textsc{ldne} is that it does not require temporal 
data, which is not feasibly attainable in many study systems \citep{Waples:2008}.

Choice of estimation method can greatly benefit from knowledge of a system's underlying 
demography. There is no one best method across all scenarios, so designing a study suitable to 
a method that is predicted to work best under certain demographic conditions can increase 
accuracy in estimating $N_e$. Future study may also be able to test whether a more 
complex weighting of the different estimators may improve upon composite estimates.

The ease of implementing an estimation method is also worth considering. \textsc{ldne} 
and \textsc{mlne} were two of the simplest programs to implement from a user's point of view. 
\textsc{neestimator} is very user-friendly, produces its estimates in a matter of seconds to
minutes, and is also one of the more actively maintained programs. \textsc{mlne} by default 
performs its moment and likelihood methods together and runs in a matter of minutes to hours. 
Initial setup of \textsc{mlne} may be slightly more daunting to users less familiar with 
working with various software as it requires some additional steps to compile on a Linux 
or Unix-based machine. Likewise, \textsc{tmvp} comes with difficulties in its implementation 
in that it can be difficult to find a set of run parameters that function properly to allow 
convergence of the analysis, run times are on the order of days, and there is an additional 
postprocessing step to acquire the $N_e$ estimate from its output.

In conclusion, complex demographic histories make estimating $N_e$ difficult, 
but in terms of accuracy and performance, \textsc{mlne} and \textsc{ldne} have been shown 
to be the best existing methods to estimate contemporary $N_e$ of a population, 
with one method or the other being better depending on knowledge of migration rates and 
approximate idea of the magnitude of size of the population at hand.

%%% Local Variables:
%%% TeX-master: "thesis"
%%% TeX-PDF-mode: t
%%% End:



