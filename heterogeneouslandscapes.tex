\chapter{Genetic architecture and landscape heterogeneity interact to determine adaptation during species range expansion}%The role of genetic architecture and environmental gradients in adaptation during range expansion}
\label{chap:heterogeneouslandscapes}

\section{Abstract}


\section{Introduction}

% speciation, local adaptation, range expansion are all about adaptation to novel environments
%	in particular, adaptation in the face of gene flow

% it is still today a major question in evolutionary biology what alleles contribute to local adaptation
% and equally unknown (?) and much more hotly debated is how many genes are needed for speciation to occur? few or many? or a whole genome? or one magic trait?

% by examining range expansion over different types of heterogeneous environments and with different underlying genetic architectures, we investigate the ability for populations to colonize new habitat and adapt in order to undergo a species range expansion


%discuss the question of adaptation broadly - it's important in many aspects of evolution
The question of why species range limits exist has pervaded evolutionary biology for decades. Understanding populations' abilities to adapt to novel environments is key to this process, and is a major area of study across evolutionary biology in studies of speciation, local adaptation, invasive species, and conservation biology. 
Populations are known to locally adapt within their species range to survive across a range of disparate environments \citep{Kawecki:2004}. On a more extreme scale, populations may eventually span such dissimilar environments that they reach a point of sufficient differentiation to produce new, isolated species adapted to different environments \citep{Rundle:2005, Doebeli:2003}. Adaptation to new environments determines the capacity of invasive species to thrive \citep{Prentis:2008}, and the potential for success of new genetic diversity from attempts of assisted migration \citep{Aitken:2013}. 
Regardless of whether adaptation results in the formation of new species or differentiated local populations, it remains a major question in evolutionary biology as to which and how many genes contribute to local adaptation. 

% talk about genetic architecture
Many studies, both empirical and theoretical, examine genetic architecture and its role in the ability to adapt to a novel environment \citep{Yeaman:2015, Yeaman:2011, Carroll:2001, Holloway:1990, Peichel:2001, Bratteler:2006, Schiffers:2014}. It is known that a sufficient level of genetic variation, $V_G$, is necessary to provide the alleles required for adapting to a new environment. It was previously thought that many alleles of too small an effect may be swamped by gene flow before being able to contribute to sufficient local adaptation, but as shown by \citep{Yeaman:2015}, this effect is possible through transient processes subject to random genetic drift when rates and sizes of mutations and genetic redundancy are sufficient (though detecting the presence of such alleles remains a difficult task). It is clear however, that fewer loci of larger effect can contribute best to adaptation, which eventually lead to the building up of linked clusters of locally adapted alleles, often referred to as genomic islands of divergence \citep{Feder:2010}. Even when small effect alleles may contribute to adaptation, larger effect alleles build up and contribute to adaptive differences among populations \citep{Yeaman:2011}. As \citet{Renaut:2013} have shown in sunflower, genetic architecture plays a larger role in contributing to genomic divergence than does the geography of speciation. %make sure to go back and see what they mean by this "geography"
The details of genetic architecture are thus vital in studies of adaptation, and should not, nor need not, be ignored in today's era of genomic technologies. % eh, maybe remove the last part here, I am only doing a simulation study after all

% talk about landscape heterogeneity - in broad terms
Whether a certain genetic architecture will successfully contribute to adaptation in a new environment will also depend largely on the environmental difference at hand. There are clearly biological limits that exist in terms of adapting to environments of extreme difference in optimal phenotype, but within the realm of biological reality, there may still be limits to the change in environmental optimum to which a given genetic architecture allows adaptation. In the real world, environments are heterogeneous, and they change in multifarious ways over space, from gradual to sudden, and of minor or major magnitudes of change. \citet{Schiffers:2014} have investigated the role of adaptation across patches of heterogeneous environment, but it has not yet been investigated how the degree of change in these optima will impact the outcome of adaptation during range expansion.

MORE ABOUT WHAT TYPES OF HET LANDSCAPES EXIST BIOLOGICALLY

%The potential for genetic architecture underlying adaptation to interact with the environment to which populations are adapting is however less well understood. \emph{somewhere in here need to link better to the fact that I'm focusing on range expansions} %%

% talk about landscape heterogeneity in terms of range expansions
When considering environmental heterogeneity in terms of adaptation during range expansions, there are several important genetic processes that can lead to interesting hypotheses for adaptation across the landscape, making range expansions ideal for studying the combination of these effects. 
% Species exist on environmentally heterogeneous landscapes, and the degree of this heterogeneity can vary greatly within species ranges.
The characteristics of the environment available for a species to expand into plays an important role in the ability of populations to expand \citep{Aguilee:2012, Barton:2001, Pease:1989}. Increasing habitat heterogeneity, and thus heterogeneity in selective pressures, interacts with gene flow to influence the adaptive abilities of peripheral populations, and can in some cases inhibit local adaptation in peripheral populations \citep{Slatkin:1987,Kirkpatrick:1997, Ronce:2001}. During a range expansion, small founder populations at the margin experience reduced effective population sizes and reduced efficacy of selection. When this process repeats over the course of an expansion, the process of gene surfing \citep{Klopfstein:2006} can increase the presence of deleterious alleles and cause an expansion load at the range margin \citep{Peischl:2013}. This process is examined in more detail by Gilbert \emph{et al.} (\color{red} \footnotesize hopefully submitted \normalsize \color{black}). %or remove this sentence and just talk about dispersal?
Due to reduced density at range margins, these populations experience gene flow at much higher proportions than any populations existing within the denser core os the species range. This asymmetric migration can behave in two disparate ways: introducing new genetic variation needed for further adaptation in an otherwise genetically depauperate population, or swamping locally adaptive alleles with foreign maladaptive alleles thus preventing further adaptation. Which of these processes occurs will depend on the details of migration rates, mutation rates, population sizes, and the degree to which the environment differs across the species range.

Previous theoretical studies have examined many of these aspects and their role in local adaptation at range margins during the expansion process. Higher gene flow has been shown to introduce more maladaptive alleles from greater distances and thus be more detrimental to the local population \citep{GarciaRamos:1997}. While others have shown the effect of evolutionary rescue from migration when marginal populations have small effective population sizes and behave as sinks \citep{Holt:1997,Gomulkiewicz:1999,Ching:2012}. Alternatively, if selection is strong enough or migration reduced enough, this can prevent the establishment of foreign alleles in a population allowing it to adapt sufficiently to local conditions \citep{Ronce:2001}. 

Foundational work in the field showed that on a linear environmental gradient, increasing steepness of the change in environmental optimum eventually leads to maladaptation at the margin and eventual extinction of the species \citep{Kirkpatrick:1997}. %Improving upon this study by allowing realistic and evolving genetic variance showed no limits to adaptation at a range margin, and a wider range of results when investigating temporally changing environments and varying population densities \citep{ polechova
Further investigations have improved upon this and shown the effects of density dependent selection \citep{GarciaRamos:1997} as well as stochastic effects on low density \citep{Bridle:2010}, temporal changes in environments \citep{Pease:1989}, differing dispersal parameters \citep{Aguilee:2012}, evolution of genetic variance \citep{Barton:2001,Polechova:2009}, and the impacts of genetic drift  \citep{Polechova:2015} on adaptation or the lack thereof at range margins. %\citet{Case:2000} also incorporated species competition, under which they found similar results only with a shallower environmental gradient needed to limit range expansion. Some of these models use the same linear gradient of an environmental optimum changing at a constant rate over continuous space, while others simulate adaptation to differing environmental optima across discrete space (habitat patches) within which  individuals of a patch possess no explicit spatial designation. The results of these discrete environment models also show the potential for species to either adapt and persist, exist as non-adapted generalists, or go extinct.
%  **** include this and improve on the summary of other studies? --> One may predict that a given difference in optimum phenotype can be attained equally from few loci of large effect or many of small effect, but such a process will clearly rely on important details of migration, mutation, selection, and genetic drift. (cite someone here?) Strong asymmetric migration can swamp locally beneficial alleles and prevent adaptation (cite Kirkpatrick and barton, and others). Weak selection and strong genetic drift can prevent beneficial alleles from establishing (true??). And mutation may drive the presence of genetic variation needed to allow adaptation to novel conditions.
These studies have all largely investigated adaptation in models of two discrete habitat patches of different environmental optima or across linear gradients in continuous space. Yet both of these situations differ largely from biological reality, where environmental changes exist in some combination of these two layouts: contiguous space with varying levels of gradual or abrupt environmental changes distributed patchily across space. %may occur more or less frequently in space and to a greater or lesser degree of difference in phenotypic optimum. 
These differences in landscape play a key role in the process of populations adapting because of their potential to interact with the effects of evolutionary processes at the range margin. 

% set up our hypotheses/goals
When discrete habitat types exist contiguously across space, will the dynamics of the system reflect those of the continuous or the discrete models? %The combination of such environmental landscapes however, has not been sufficiently explored. 
\citet{Barton:2001} suggested the ability of populations to persist across environments of accelerating gradients, but in depth examination of different types of environmental heterogeneities is warranted. Because discrete models have as of yet not extended past two patches, it is unclear what effects a third or additional patches may have on evolution and adaptation within the system when space is explicit within each patch. With larger environmental differences in optima, will only large effect mutations be able to contribute to local adaptation and small effect alleles still swamped at the range margin?

% this project's hypotheses
In this study, we test if different genetic architectures and landscape heterogeneities can interact and affect the outcome of adaptation during a species range expansion, and potentially even limit a species range expansion. We compare our results to predictions from \citet{Barton:2001} and the results of \citet{Bridle:2010} and \citet{Schiffers:2014} to assess the impacts of our further explorations into genetic architecture and environmental heterogeneity. 
% how we're going to test those hypotheses
To address this question, we simulate a spatially explicit and discrete two-dimensional landscape across which populations expand their species range and adapt to newly encountered environments. We model a range of genetic architectures for a quantitative trait to test for differences in ability to adapt to environments. We also compare this across environments that differ in their degree of environmental heterogeneity from few to many locations of change in environmental optima, as well as in terms of the magnitude of change for each of these optima. Examining the rate of range expansion and the time required for populations to colonize and adapt in new environments provides insight into the role of both genetic architecture and environmental heterogeneity in determining the success or failure of adaptation. Such results are highly informative for the field of evolutionary biology, and largely applicable in predicting the fate of many species across the globe requiring the ability to adapt in the face of disturbance and shifting species ranges due to climate change and other anthropogenic environmental perturbations.


%<><><><><><><><><><><><><><><><><><><><><><><><><><><><><><><><><><><><><><>

% marginal populations thus often exist only as a sink, sustained by immigration until sufficient local adaptation is achieved. % and recovery from expansion load? -- refer to Ch 4 here, i.e. refer to our group project paper -- or just talk less about load since I'm not studying it in this paper
% When the population shifts to be a self-sustaining source, it escapes the effects of drift, and provides emigrants for further colonization and range expansion. This shift from sink to source also includes a shift away from asymmetric migration previously swamping locally adaptive alleles with foreign maladaptive alleles.

%Local adaptation within a species not only benefits populations in terms of fitness but also has repercussions for a species' global range. A species' range is defined in some cases by sharp visible boundaries, such as distinct geographical changes in landscape features, or equally understandable ecological reasons such as competitor occurrence or nutrient limitations. In instances where no such clear reason for a range edge exists, understanding what forces limit expansion and adaptation at the edge of a range becomes an evolutionary question \citep{Bridle:2007,Kawecki:2008,Excoffier:2009}. Barriers to dispersal may prevent expansion past the edge, however gene flow can also act to inhibit local adaptation in peripheral populations \citep{Slatkin:1987,Kirkpatrick:1997}. This, in combination with different selective forces from environmental gradients across the landscape, may prevent expansion. The characteristics of the environment available for a species to expand into plays an important role \citep{Aguilee:2012,Barton:2001,Pease:1989}. Increasing habitat heterogeneity, and thus heterogeneity in selective pressures, interacts with gene flow to influence the adaptive abilities of peripheral populations \citep{Ronce:2001}.

% Under one of the first models to examine the question of adaptation across species ranges \citep{Kirkpatrick:1997}, it was shown that three different scenarios could result: species extinction, a limited species range, or unlimited expansion of a species. Other studies have examined scenarios similar to this model and incorporated more realistic simulation methods that were previously not investigated: density dependent selection \citep{GarciaRamos:1997,Gomulkiewicz:1999} as well as stochastic effects on low density \citep{Bridle:2010}, temporal changes in environments \citep{Pease:1989}, differing dispersal parameters \citep{Aguilee:2012}, and evolution of genetic variance \citep{Barton:2001,Polechova:2009}. \citet{Case:2000} also incorporated species competition, under which they found similar results only with a shallower environmental gradient needed to limit range expansion. Some of these models use the same linear gradient of an environmental optimum changing at a constant rate over continuous space, while others simulate adaptation to differing environmental optima across discrete space (habitat patches) within which  individuals of a patch possess no explicit spatial designation. The results of these discrete environment models also show the potential for species to either adapt and persist, exist as non-adapted generalists, or go extinct.

%Equally relevant to studying adaptation in a local adaptation or speciation context, is the study of adaptation during a species range expansion. As a species range expands, it encounters novel habitat to which it must adapt. Previous studies have shown that it is the ability of populations at range edges to adapt to these new environments that can determine the ability of a range to expand (citations). Not unlike parapatric speciation, this adaptation to new environments must occur in the presence of gene flow. Asymmetric gene flow at a range edge, however, can cause migration load that is particularly strong when an edge population is still a sink and requiring the influx of individuals to persist.

%Species range limits and the capacity for range expansion has long been a question in evolutionary biology. A species' range is defined in some cases by sharp visible boundaries, such as distinct geographical changes in landscape features, or equally understandable ecological reasons such as competitor occurrence or nutrient limitations. In instances where no such clear reason for a range edge exists, understanding what forces limit expansion and adaptation at the edge of a range becomes an evolutionary question \citep{Bridle:2007, Kawecki:2008, Excoffier:2009}. Barriers to dispersal may prevent expansion past the edge, however gene flow can also act to inhibit local adaptation in peripheral populations \citep{Slatkin:1987, Kirkpatrick:1997}. This, in combination with different selective forces from environmental gradients across the landscape, may prevent expansion. 


%<><><><><><><><><><><><><><><><><><><><><><><><><><><><><><><><><><><><><><>



\section{Methods}

%\subsection{Simulation Model}

We model a species range expansion over heterogeneous environments with individual-based, forward time simulations in the program \textsc{nemo} \citep{Guillaume:2006}. We vary two main parameters of interest: the magnitude and heterogeneity of change in environmental optimum in space and the genetic architecture underlying the quantitative trait that allows adaptation to the local environment. We also briefly compare these effects across two cases of small and large average dispersal distances. Our goal is to understand the impact and interactions of genetic architecture and the landscape heterogeneity over which individuals exist in terms of the outcome of range expansion and population adaptation.

We implement a modified version of \textsc{nemo} \citep{Guillaume:2006}, described in Gilbert \emph{et al.} (\Chapref{expansionload}), which allows for a large-scale, two-dimensional species range to be modeled with explicit space. This expands greatly upon existing 2-patch models with non-explicit space within each landscape patch and approaches a model of continuous space. Individuals are monoecious (hermaphroditic), obligately outcrossing diploids which possess a quantitative trait under balancing selection for the local environmental optimum. We initiate the population in a limited region for a burn-in to migration-mutation-selection equilibrium, after which populations expand into empty landscape of various types of heterogeneous environmental optima. Generations are non-overlapping and life cycle events occur in the order of breeding, dispersal, viability selection, and population regulation. We monitor the populations' abilities to spread across the landscape, and when successful, compare these scenarios in terms of expansion speed and fitness as they encounter environments of differing environmental optima to which they must adapt in order to further expand.


% now go into all the gory details:
\subsection{Landscapes}

To approximate continuous space and maintain a spatially explicit landscape with discrete populations, we implemented a large spatial grid of habitat patches in \textsc{nemo}. This bridges the gap between existing two-patch landscape models, where existence within either of these two patches is not spatially defined, and models of continuous space. The landscape is a rectangular grid of $40\times2000$ cells with a defined phenotypic optimum and carrying capacity for every cell in the landscape. 
This greatly builds upon the model of \citet{Schiffers:2014} which examined a grid of $16\times64$ cells. We term the leftmost $40\times40$ cells of the landscape the core, which is where the burn-in period occurs. After burn-in, expansion proceeds along the long ($x$) axis of the landscape into the remaining $40\times1960$ cells. The core consistently possessed a constant optimum phenotypic value of 0 to ensure that the population was well-adapted before populations expanded onto environmental gradients. In all cases, the environmental optimum only changed in one dimension over the axis of expansion (x-axis) and was constant over cross-sections of the landscape (y-axis). Terms for describing these landscape and other parameters are defined in Table \ref{tab:params}.

We examined several types of environmental heterogeneities across landscapes. These range from a linear gradient to only two patches each with a different local environmental optima, as well as combinations in between with multiple patches of constant environmental optima interspersed by sharp changes in the environment. This heterogeneity is representative of a multitude of possibilities in the real world. Many landscapes are patchy over space, for example patches of forest or meadow, and human-induced habitat fragmentation has only increased over time (citations). Similarly, many environments can change sharply in space, for example moving from forested habitat to above the timberline on elevational gradients. %Such patchy landscapes can exist on environmental gradients such as latitudinal (or altitudinal) temperature change.
Our modeled landscapes of constant environment interspersed among increases in environmental optima are thus not unrealistic biologically. 

To define each type of heterogeneous landscape we modeled, we describe three parameters: the number of changes in environmental optima over the landscape, the magnitude of each of these changes, and the total gradient value across the landscape, which can be calculated from the previous two values. We refer to each change in environmental optima on the landscape as ``steps" since the optimum is constant within a patch then changes in magnitude to the new optimum on a subsequent step. The number of changes, or steps, on the landscape is equivalent to describing the width of each step, since the landscape size is constant over all simulations and no changes occur perpendicular to the axis of expansion. The magnitude of change at a given step is referred to as the step's height, $h_{step}$ (Figure \ref{fig:stepschematic}). Therefore, a linear gradient would be one with $1959$ steps, each with a width of one cell. 

\begin{figure}[h]
\centering
\makebox[\textwidth]{
        \includegraphics[width=0.9\linewidth]{Figures/StepSchematic_Figure.pdf}}
\caption[Schematic of step landscapes.]{Schematic of simplified step landscapes, where step height ($h_{step}$) is the magnitude of change in environmental optimum at one location on the landscape and step width is the number of units in the x-direction contained within a given area of the same environmental optimum. Landscapes depicted with solid lines have the same step height across the landscape, but different total gradients, while the dashed and gray lines show how different numbers of steps of different height can make up a landscape with the same overall gradient. The landscape core in the first 40 units of the landscape is constant, and the linear gradient is where there is a step at every unit x on the landscape outside of the core.}
\label{fig:stepschematic}
\end{figure}


% maybe move to results
We first model linear gradients to determine the critical value, $b_{crit}$ for which expansion could no longer occur. From this we then break down this linear gradient into steps of environmental change to test at what point this prediction of $b_{crit}$ fails to identify successful expansions. This held the total gradient value constant, but changed the frequency of steps and thus also each step's height. To understand why this $b_{crit}$ prediction fails, we then investigate the role of step height alone on a two-patch landscape, and then elucidate the interaction of step height with step width in allowing adaptation and expansion to proceed across the landscape. All results present  the average value for a given parameter across the $y$-axis of the landscape for any given landscape location $x$. 

%%To first determine the factors underlying adaptation to a single, new environment, landscapes with one step were tested for $h_{step} = 5, 7.5, 10, 12.5,$ and $15$. In these cases, $h_{total}$ = $h_{step}$, and essentially mimics a spatially explicit 2-patch model. % For the same total change in optimum over the landscape, but at the other extreme of heterogeneity, i.e. being linear, populations had no difficulty in expanding to fill the landscape when $h_{total} \sim 0.0025, 0.005,$ and $0.0075$.
%%Upon determining the outcomes of success or failure to expand across this single step, we then simulated landscapes with increasing frequencies of steps, while holding $h_{step}$ constant at either 5 or 10, within each simulation. Step frequencies ranging from 4 to 1,959 (linear) steps per landscape were compared, which can equivalently be described as scenarios where the width of a step (the number of cross sections between subsequent steps) ranges from 392 cross sections down to a single cross section between steps on the linear gradient. Because $h_{step}$ is held constant in these scenarios, the total change in environmental optimum across the entire landscape, $h_{total}$, differs across cases of different step frequencies and is simply the product of $h_{step}$ and the number of steps on the landscape. The scenario for $h_{total}$ being held constant is not shown, since linear gradients with $h_{total} = 5, 10,$ or $15$ ($h_{step} = 0.0025, 0.005,$ or $0.0075$) posed no difficulty for any of our simulated populations to expand across and adapt (see Supplemental Figure \color{red}1\color{black}). 
% if I have time for the simulations, add in the description of non-monotonic step landscapes

% Multiple different scenarios can represent the same total change in landscape optimum (Figure 1a) while the frequency of steps in space can also result in different total changes in landscape optimum (Figure 1b).


%<><><><><><><><><><><><><><><><><><><><><><><><><><><><><><><><><><><><><><><><><><><><><><><><>
\begin{table}[h]
\centering \footnotesize
\caption{Terminology and parameter definitions for simulations}
\label{tab:params}
\begin{tabular}{lp{0.8\textwidth}l}
Term		& Explanation  \\ \hline \hline
Cell		& One unit of space on the landscape grid. 40 cells across the y-axis constitute a landscape cross-section. The entire landscape is $40\times2000$ cells.			\\ \hline
Core		& The leftmost $40\times40$ landscape cells. The population is limited to this region during the burn-in period. Environmental optimum is constant within the core.	\\ \hline
Step		& Used to describe an instance of a change in phenotypic optimum on the landscape.															\\ \hline
$h_{step}$ & The magnitude of change in phenotypic optimum across a step in the landscape. In other words, the height of a step.									\\ \hline
$h_{total}$ & The total magnitude of change in phenotypic optimum across the entire landscape. In other words, the sum of all step heights.							\\ \hline
Step width	& The number of landscape cross sections between subsequent changes in environmental optimum. In other words, the number of cross sections between each occurrence of a step in the landscape.	\\ \hline
\emph{K}	& Carrying capacity of a cell. Population regulation occurs at the level of cells. $K = 5$ in all cases.													\\ \hline
$\sigma_{breed}$ & One standard deviation of the Gaussian kernel describing the area of the breeding window. Individuals can find mates within cells contained by the breeding window. $\sigma_{breed} = 0.5\times$cell width in all cases. \\ \hline
$\sigma_{disperse}$ & One standard deviation of the Gaussian kernel describing the area of potential dispersal. Offspring can disperse to cells within the dispersal kernel. $\sigma_{disperse} = 2\times$cell width or $4\times$cell width as described in the text.                                          
\end{tabular}
\end{table}
%<><><><><><><><><><><><><><><><><><><><><><><><><><><><><><><><><><><><><><><><><><><><><><><><>








\subsection{Breeding \& Dispersal}
The large landscape grid allows regulation at a fine scale across the landscape, which is what allows our simulations to approach a continuous spatial model. Each landscape cell has a carrying capacity set at $K = 5$, which allowed for specification of a fine scale landscape to underly larger populations. We maintain larger effective population sizes through breeding and dispersal which both occur over regions of nearby cells. This creates a neighborhood size of approximately 250 individuals, calculated from \citet{Wright:1946}. 

The breeding window we introduced into \textsc{nemo} defines a given radius of cells around a focal cell from which an individual can find a potential mate during the breeding life cycle stage and is described in detail in Gilbert \emph{et al.} \Chapref{expansionload}. Because modeling a range expansion implies that there will often be lone colonists on the expansion front, this addition of a breeding window increases biological realism by allowing lone colonists on the range front to still potentially find a nearby mate, even if they are the sole inhabitant of their cell. This most closely resembles obligately outcrossing plants, who may receive pollen from nearby mates, but could also represent animals that search for mates nearby then return to their home territory. The probability of finding a mate is described by an approximate bivariate Gaussian defined by one standard deviation of a Gaussian distribution, $\sigma_{breed}$, where $f(x,y) \propto \exp{[-(\frac{\Delta x^2}{2\sigma_{breed}^2}+\frac{\Delta y^2}{2\sigma_{breed}^2})]}$ gives the distance traveled to search for a mate in a given direction $x$ and $y$. We discretized these probabilities by integrating the probability over each cell, and then weighted these probabilities by the number of potential mates in each cell within the breeding window. $\sigma_{breed}$ was set at one half of a cell length, and the maximum search radius for a mate was limited to $4\times\sigma_{breed}$. This produced a breeding window containing the 13 cells: the focal cell and its 12 surrounding cells. Individual fecundity was drawn from a Poisson distribution with mean 4 to determine the number of offspring to be produced.

Dispersal occurred similarly to breeding, where a dispersal kernel was defined for forward rates of migration and discretized over cells on the landscape. This distribution was defined by $\sigma_{disperse} = 2\times$ cell size for the majority of simulations. Where specified, a larger dispersal kernel of $\sigma_{disperse} = 4\times$ cell size was implemented. An R wrapper package was created to calculate breeding and dispersal kernels, and to discretize these probabilities across the landscape, and is available online at https://github.com/kjgilbert/aNEMOne. C++ code for the modified version of \textsc{nemo} with a breeding window is available online at https://github.com/kjgilbert/NemoDispersalKernel.
 
 
 
\subsection{Genetic Architecture}
The quantitative trait, $z$, is controlled by 100 freely recombining loci. We held the total mutational variance, $V_m$, available to the system constant at $10^{-2}$, but varied the underlying genetic architectures contributing to the mutational variance. The number of loci, $L$, was also held constant at 100, thus to satisfy $V_m = 2 L \mu \alpha^2$, we varied the per locus mutation rate, $\mu$ and the variance in mutational effect sizes, $\alpha^2$. 
Four genetic architecture regimes were investigated as follows. First, we employed one with small effect mutations and a higher mutation rate: $\mu = 10^{-2}$ and $\alpha^2 = 0.005$. Second and third intermediate regimes set $\mu = 10^{-3}$ and $\alpha^2 = 0.05$, and $\mu = 10^{-4}$ and $\alpha^2 = 0.5$. The fourth regime allowed for the largest effect mutations to occur proportionally more often, yet exhibit less frequent mutations overall where $\mu = 10^{-5}$ and $\alpha^2 = 5.0$. 



\subsection{Viability Selection}
Selection occurs only on survivorship throughout the simulations. Selection is stabilizing for the phenotypic optimum, with fitness of an individual is defined by $\omega_z = exp[-\frac{(z-z_{opt})^2}{2V_S}]$, where $z$ is the quantitative trait value, $z_{opt}$ is the optimum phenotypic trait for a given cell on the landscape, and $V_S$ defines the width of the fitness function. We set $V_S = 7.5$, and environmental variance, $V_E$, was set to 1. This follows from choice of $V_S$ and $V_E$ in Gilbert \emph{et al.} (\Chapref{expansionload}) as was derived from \citet{Kingsolver:2001} and \citet{Johnson:2005}.
 
 
 

% A 1-dimensional probability distribution is drawn, defined by one standard deviation of a Gaussian distribution $\sigma$, which is then multiplied by itself to create a 2-dimensional kernel. This is reduced to include only probabilities equal to or higher than the smallest probability from the 1-dimensional kernel, and is then renormalized. 


%Figure 1 - Visualizations in 1-dimension of heterogeneously changing environmental gradients. Each horizontal portion is referred to in the text as a step. $h_{step}$ is a measure of the vertical distance between steps while $h_{total}$ is the total change in environmental optimum from one end of the landscape to the other. Panel A shows gradients that have the same overall slope, b ($h_{total}$ = 200), but different magnitudes of change in the local optimum, $h_{step}$. Panel B shows gradients that have different overall slopes, b, but the same magnitude of change in the local optimum per step ($h_{step}$ = 4).




\section{Results}

\subsection{Expansion on a linear gradient}

We investigated the critical steepness of a linear gradient first, as a comparison to existing literature.
...

when is b too steep on a linear gradient?

	might need to calculate b as the gradient within a sigma or something like that...

what value of b is it above which no expansion occurs

even when expansion occurs on lower values of b, the lower the b, there are still differences

speed of expansion
total pop size across the landscape
fitness?

\subsection{Expansion over heterogeneous gradients}

total h = 15 runs expand fine

one step h = 15 runs never expand (almost, except that one case)

where on the steppiness continuum does the b prediction from linear gradients fall apart?

total h is still the same in all these cases

\subsection{Expansion across one change in environmental optimum}

why does the steppiness matter?

step height

	one step analyses show height matters
	need the mutations or standing variation to adapt


step width

	but making it over one step of a given height doesn't ensure that you can make it over multiple steps of that height
	too many steps of a given height again prevents expansion
	
	so step width also matters
		why?
		migration load from other steps
		need recovery that mig load prevents (local adaptation not possible)
		
		ratio of step width to step height that is the critical value
		or ratio of step width to total gradient that is the critical value?

\subsection{Impact of genetic architecture}

above results were for only one type of genetic architecture, but these results quantitatively and qualitatively change for different types of genetic architecture

go through all of the above again for the 3 other gen archs:
	linear gradient
	steppy gradient
	1 step cases

how do values of b-crit and the step-ratio-thing differ in these cases?


why does genetic architecture matter?

	looking at 1 step cases, can see that expanding over them depends on the distribution of mutations and\/or the standing genetic variance
	larger effect mutations at fewer loci contributed to adaptation across the step in one step cases
	
\subsection{Impact of dispersal distances}

larger dispersal distances make everything more difficult (make sure this is true)

because more migration load

what are the new b-crits?


\section{Discussion}



%\section{Acknowledgements}
%We would like to thank Fred\'eric Guillaume for help with \textsc{nemo} and members of the Whitlock lab for comments and feedback.


%%% Local Variables:
%%% TeX-master: "thesis"
%%% TeX-PDF-mode: t
%%% End:
