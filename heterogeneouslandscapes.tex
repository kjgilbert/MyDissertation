\chapter{The role of genetic architecture and environmental gradients in adaptation 
during range expansion}
\label{chap:heterogeneouslandscapes}

\section{Abstract}


\section{Introduction}

Species range limits and the capacity for range expansion has long been a question in evolutionary biology. 
A species’ range is defined in some cases by sharp visible boundaries, such as distinct geographical 
changes in landscape features, or equally under- standable ecological reasons such as competitor 
occurrence or nutrient limitations. In instances where no such clear reason for a range edge exists, 
understanding what forces limit expansion and adaptation at the edge of a range becomes an evolutionary 
question \citep{Bridle:2007, Kawecki:2008, Excoffier:2009}. Barriers to dispersal may 
prevent expansion past the edge, however gene flow can also act to inhibit local adaptation in 
peripheral populations \citep{Slatkin:1987, Kirkpatrick:1997}. This, in combination with 
different selective forces from environmental gradients across the landscape, may prevent expansion. 
The characteristics of the environment available for a species to expand into plays an important 
role \citep{Aguilee:2012, Barton:2001, Pease:1989}. Increasing habitat heterogeneity, and 
thus heterogeneity in selective pressures, interacts with gene flow to influence the adaptive 
abilities of peripheral populations \citep{Ronce:2001}.

Local adaptation within a species not only benefits populations in terms of fitness but also has 
repercussions for a species' global range. A species' range is defined in some cases by sharp visible 
boundaries, such as distinct geographical changes in landscape features, or equally understandable 
ecological reasons such as competitor occurrence or nutrient limitations. In instances where no 
such clear reason for a range edge exists, understanding what forces limit expansion and adaptation at 
the edge of a range becomes an evolutionary question \citep{Bridle:2007bi,Kawecki:2008gv,Excoffier:2009em}. 
Barriers to dispersal may prevent expansion past the edge, however gene flow can also act to inhibit local 
adaptation in peripheral populations \citep{Slatkin:1987uf,Kirkpatrick:1997jo}. This, in combination with 
different selective forces from environmental gradients across the landscape, may prevent expansion. The 
characteristics of the environment available for a species to expand into plays an important role 
\citep{Guilee:2012hk,Barton:2001ti,Pease:1989wj}. Increasing habitat heterogeneity, and thus heterogeneity 
in selective pressures, interacts with gene flow to influence the adaptive abilities of peripheral popula
tions \citep{Ronce:2001vpa}.

Much theoretical work has examined the role of these factors in local adaptation at range 
edges \citep{GarciaRamos:1997,Ibrahim:1996,Gomulkiewicz:1995,Atkins:2010}. There are 
several key points in the theoretical foundations of this topic. First, across a species range, 
density of populations varies. In some cases the center of the range is most dense and this 
density decreases towards the edge, commonly referred to as the `abundant centre hypothesis' 
\citep{Brown:1984}. Though not present in all species \citep{Sagarin:2006}, this pattern 
is widely documented \citep{Antonovics:1976,Yeh:1979,ECKERT:2008}. %% check on this Yeh paper
The result of such differential densities is to create asymmetries in gene flow across a 
species range, where central populations contribute more to the periphery while peripheral 
populations contribute less to the center. This asymmetry can lead to swamping of local alleles 
by foreign alleles at the edge. And when central alleles are adapted to different environmental 
conditions, they may be maladaptive at the range edge, resulting in peripheral populations 
exhibiting a reduction in adaptation to the local optimum and decreasing mean fitness the 
further from the center one moves. This has been shown to prevent range expansion under steep 
changes in environmental gradients across space \citep{Kirkpatrick:1997}.

%describe the tests that have already been done of K & B
This effect clearly depends on levels of gene flow and selective forces at the edge. Higher gene flow will introduce more maladaptive alleles from greater distances and thus be more detrimental to the local population \citep{GarciaRamos:1997uu}, yet in some cases this may act as a sort of evolutionary rescue when edge populations have small effective population sizes and behave as sinks \citep{Holt:1997tp,Gomulkiewicz:1999wj,Ching:2012jk}. Such gene flow may provide the necessary genetic variance for a population to survive. Alternatively, if selection is strong enough or migration reduced enough, this can prevent the establishment of foreign alleles in a population allowing it to adapt sufficiently to local conditions \citep{Ronce:2001vpa}.

The above points have been investigated by models of continuous space with a linear change in environmental gradient \citep{Pease:1989wj,Kirkpatrick:1997jo,GarciaRamos:1997uu,Case:2000ci,Polechova:2009ir,Bridle:2010km} or by simulating discrete habitat patches of differing environmental optima \citep{Guilee:2012hk,Ronce:2001vpa,Gomulkiewicz:1999wj}. In reality however, environments exist in some combination of these two layouts: contiguous space with varying levels of gradual or abrupt environmental changes distributed patchily across space. \textbf{In this project, I aim to examine the consequences of varying amounts of environmental heterogeneity on the outcome of local adaptation across a species range, and particularly at range edges where lack of adaptation may prevent range expansion.}

Under one of the first models to examine the question of adaptation across species ranges \citep{Kirkpatrick:1997jo}, it was shown that three different scenarios could result: species extinction, a limited species range, or unlimited expansion of a species. Other studies have examined scenarios similar to this model and incorporated more realistic simulation methods that were previously not investigated: density dependent selection \citep{GarciaRamos:1997uu,Gomulkiewicz:1999wj} as well as stochastic effects on low density \citep{Bridle:2010km}, temporal changes in environments \citep{Pease:1989wj}, differing dispersal parameters \citep{Guilee:2012hk}, and evolution of genetic variance \citep{Barton:2001ti,Polechova:2009ir}. \citet{Case:2000ci} also incorporated species competition, under which they found similar results only with a shallower environmental gradient needed to limit range expansion. Some of these models use the same linear gradient of an environmental optimum changing at a constant rate over continuous space, while others simulate adaptation to differing environmental optima across discrete space (habitat patches) within which  individuals of a patch possess no explicit spatial designation. The results of these discrete environment models also show the potential for species to either adapt and persist, exist as non-adapted generalists, or go extinct.

The combination of such environmental landscapes however, has not been sufficiently explored. When discrete habitat types exist contiguously across space, will the dynamics of the system reflect those of the continuous or the discrete models? \citet{Barton:2001ti} suggested the ability of populations to persist across environments of accelerating gradients, but the topic has not been explored in depth. Because discrete models have as of yet not extended past two patches, it is unclear what effects a third or additional patches may have on evolution and adaptation within the system when space is explicit within each patch. \textbf{I propose incorporating increasingly realistic environments in continuous space over which adaptation and evolution of a species may be investigated.}
%% Dick Gomulkiewicz said at EvoWibo 2012 that Bob Holt may be doing a 3-island model of this stuff; check back on this!
%% Amy -- you haven't explained whether/how there are different results from models with discrete vs continuous habitats (or why you would expect the difference to matter), so this contrast is not motivated for the reader.  

Range expansions have been studied…
But range limits have not been shown in theory…
But those sims have lots of assumptions…
Want to bring together the disconnect between continuous space on linear gradients and non-spatially 
explicit patch models.

If we look at landscapes that change heterogeneously over space instead of linear gradients, 
what happens? Can this explain or inform real life range limits?
Do find limits
	How do they compare to 2-patch models (gomulkiewicz and holt)
	How do the overall gradient values compare to K\&B 1997 \& those follow-ups


\section{Methods}

Simulation Model
We model a range expansion on a two-dimensional landscape using individual-based simulations 
from a modified version of Nemo (Guillaume and Rougemont 2006). Individuals are monoecious 
diploids (hermaphrodites) that possess a quantitative trait with 100 underlying loci. 
Quantitative alleles mutate at rate 0.0001 unless otherwise stated and the mutation variance 
parameter was varied as described in the results. Free recombination was allowed, VS on the 
trait was set to 7.5 and an environmental variance of one was allowed. Each cell on the 
landscape (see below) is given a value that sets the optimum phenotype at that location 
to which the quantitative trait adapts under balancing selection.
The life cycle consisted of four events: breeding, dispersal, viability selection, and regulation. 
Because we sought to simulate individuals that exist on spatially explicit landscapes, we modified 
the Nemo code as follows (code available at https://github.com/kjgilbert/NemoDispersalKernel). 
To make the presence of an individual within a deme more spatially explicit, we considered each 
of Nemo’s “patches” to instead be a cell on the landscape grid and simulated many cells across 
the landscape. We set carrying capacity K = 5 within each cell, individual fecundity to an 
average of 4, and allowed mating to occur within a breeding window. This breeding window 
specifies the distance within which a potential mate can be found and the likelihood of 
mating with that individual based on a Gaussian kernel. Because modeling a range expansion 
implies that there will often be lone colonists on the expansion front, this addition of a 
breeding window increases biological realism by allowing lone colonists on the range front 
to still potentially find a nearby mate, even if they are the sole inhabitant of their cell. 
This most closely resembles obligately outcrossing plants, who may receive pollen from nearby 
mates, but could also represent animals that search for mates nearby and return to their home 
territory. This increases our local effective population size above the scale of cells to 
approximately    ~~~~ . The size of this breeding window is defined in the same manner as the 
dispersal kernel. A 1-dimensional probability distribution is drawn, defined by one standard 
deviation of a Gaussian distribution σ, which is then multiplied by itself to create a 2-dimensional 
kernel. This is reduced to include only probabilities equal to or higher than the smallest 
probability from the 1-dimensional kernel, and is then renormalized. Upon mating, offspring 
are then placed in the mother’s cell, after which dispersal would proceed as normal in the 
subsequent life cycle event.
Individuals existed on a landscape 2,000 cells long and 40 cells wide. A burn-in period 
of 7,500 generations occurred in the first 40 by 40 cells and after the burn-in, the 
remainder of the landscape was opened to allow range expansion to occur. Simulations 
were run for a total of 25,000 generations with 10 replicates per scenario unless otherwise 
indicated. Landscape boundaries were absorbing, so individuals who migrate outside of the 
landscape die, and individuals at the boundary have fewer cells in their potential breeding 
window. Unless otherwise stated, σbreed = 0.5 (most mates come from within your cell, but 
the 12 surrounding cells had non-negative probabilities of finding a mate), and σdisperse = 2 
for ``small dispersal kernel`` and 4 for ``large dispersal kernel`` (See supplemental text ~~~ for 
further description). 
Landscapes
We ran simulations over a range of optima values across the landscapes. The burn-in area, 
hereafter termed the core, was consistently kept at a constant optimum value to ensure that 
the population was well-adapted before populations expanded onto the environmental gradients. 
Beyond the core, several types of environmental gradients were tested. Unless otherwise stated, 
environmental optimum only changed in one dimension, the axis of expansion (x-axis), and was 
constant over cross-sections (y-axis). The slope of the overall gradient is defined by the 
parameter b, obtained by dividing the total magnitude of change in environmental optimum 
from one end of the landscape to the other, htotal, by the total length of the landscape (2,000). 
Linear gradients are those where the landscape optimum changes at every subsequent cell in the x 
direction by a constant value hstep outside of the core. Heterogeneously changing landscapes have 
constant optima within local regions of the landscape (steps) that change periodically by value hstep. 
Multiple different scenarios can represent the same total change in landscape optimum (Figure 1a) 
while the frequency of steps in space can also result in different total changes in landscape 
optimum (Figure 1b).

Figure 1 - Visualizations in 1-dimension of heterogeneously changing environmental gradients. 
Each horizontal portion is referred to in the text as a step. hstep is a measure of the vertical 
distance between steps while htotal is the total change in environmental optimum from one end 
of the landscape to the other. Panel A shows gradients that have the same overall slope, b (htotal = 200), 
but different magnitudes of change in the local optimum, hstep. Panel B shows gradients that 
have different overall slopes, b, but the same magnitude of change in the local optimum per 
step (hstep = 4).

\section{Results}

On a linear environmental gradient, overall slope of the gradient relative to mutations 
available determined whether populations would expand across the landscape. Larger variance 
in mutation effect size allowed steeper gradients to be occupied (Figure 2). Fewer replicates 
expanded over gradient values near the limit of expansion ability (Figure 2) and took more 
generations to expand.

Figure 2 - Gradient and mutation parameter combinations that do and do not expand across 
the landscape. Each plotted point represents 10 replicate simulations. Points indicated 
by + have consistent results across replicates, indicated by text in the gray shaded areas. 
Variation in expansion speed across replicates can be seen in figure S1.

However, when the environmental gradient changes heterogeneously across the landscape 
instead of linearly, overall slope no longer determines success or failure to adapt and 
expand across the landscape. There are two factors that account for adaptation and 
expansion in such cases of heterogeneously changing gradients: the magnitude of the 
change in optimum from one cell to the next and the frequency at which these changes 
occur over the landscape.
Magnitude of Optimum Change
	The effect of the magnitude of change in environmental optimum on ability to expand 
	was tested by simulations on environmental gradients that had only one location of 
	change in the optimum (e.g. Figure 1A, thickest black line), hereafter referred to 
	as a step. Such a setup may be compared to a spatially explicit 2-patch model. The 
	ability of populations to adapt to the other side of the step and expand to fill the 
	landscape was driven by total mutational availability, defined here as the mutation 
	rate times the variance in mutation effect size (Figure 3). Smaller steps are easier 
	to cross under a range of mutation parameters, but larger steps require that larger 
	effect mutations be available.

Figure 3 - Gradient and mutation parameter combinations that do and do not expand. Each 
plotted point represents 20 replicate simulations. Because the landscape optimum only 
changes at one step location in the center of the landscape, hstep = htotal. Variation 
in expansion speed across replicates can be seen in figure S2.

Frequency of Optimum Change
	The htotal values which resulted in expansion for a single step did not however result 
	in expansion for the same htotal values on a linear gradient. Therefore we sought to 
	uncover the relationship between step height and step frequency. We investigated this 
	in the context of a subset of mutational parameters, as well as two dispersal distance 
	parameters for values of h that allowed expansion over a single step.
	Figure 4 summarizes these results as measured by speed of expansion for the various 
	scenarios. Larger values of hstep again make range expansion more difficult, and larger 
	effect size mutations improve ability to adapt and expand. An interesting interaction 
	occurs between dispersal distance and ability to expand. When steps are infrequent 
	(Step width > ~20-50 depending), having a larger dispersal distance speeds up expansion, 
	since on average individuals can migrate farther every generation. Because optimum 
	changes are rarer, populations are able to adapt to their local step, as indicated by 
	higher fitness measures in Figure 5a. However, when steps become more frequent over 
	the landscape, this larger dispersal distance becomes detrimental and slows expansion.
	
Figure 4 - Each point is an average of 10 replicates. Step width = 1 is equivalent to a 
linear gradient. Rate of expansion = zero indicates that expansion did not occur.

The slowed expansion is due to several factors. When steps are too frequent in space, there 
is insufficient space for individuals to adapt to their local optimum without migrants from 
nearby steps contributing maladaptive individuals. This is shown by mean fitness values 
averaged across the center of steps decreasing as step frequency increases, and by genetic 
variance averaged across the center of steps increasing as step frequency increases (Figure 5).

Figure 5 - Mean fitness (A) and genetic variance (B) averaged across the center of steps per 
simulation, and then averaged across 10 replicate simulations per data point shown. Points 
not shown are scenarios that did not expand.

Causes (?) of success/failure to expand
As shown in figure 5, as steps increase in frequency across the landscape, genetic variance 
increases and fitness decreases. These qualities are the defining factors in determining whether 
populations are able to adapt and expand over changes in the environment (cite all the papers 
where too high genetic variance = extinction, but probably move this to discussion). Too high 
genetic variance results in maladaptation to the local optimum, decreased fitness, and eventually 
in local extinction. However, returning to the cases where there is only one step where the 
environmental optimum changes, we find that there is also a minimum amount of genetic variance 
necessary to adapt to new environments (Figure 6). When the hstep is smaller, less drastic 
mutations are required to survive across the step, so genetic variance does not become greatly 
increased (red lines Fig 6). As hstep increases, increasingly different genotypes are necessary 
to survive in the new environment, inflating genetic variance at the location of the environmental 
change as populations adapt to either side. When hstep is too large, the necessary mutations are 
not seen arising in the population, and individuals beyond the step are only a sink population 
and still adapted to the environment before the step, maintaining a low genetic variance.

Figure 6. - Genetic variance among the two cross section immediately preceding the environmental 
step (80 cells), among the two cross sections immediately after the environmental step, and 
among these four cross sections. 4 replicate runs are shown. Cases shown in red which easily 
expand, do so immediately after the burn-in ends, while cases in blue begin expanding later, 
and cases in green never succeed in expanding beyond the step.

To do: the cases in blue in fig 6 need to be scrutinized further. What mutations arise 
right when they start to expand?
	The mutations that arise in populations at the step…
		Look at their effect sizes
		Look at how many it takes




\section{Discussion}

Because previous studies (citations) found the steepness of the environmental gradient to 
determine ability to expand, as was recapitulated here, we then further investigated the underlying 
factors that impact this ability. This was done by teasing apart the effects of absolute magnitude 
of change in environmental optimum and the frequency of this change over space. As we have shown, 
on a linear gradient where expansion for a given parameter set is able to expand, the same 
parameter set can no longer expand when the environmental change is reorganized into one “step” 
in the landscape.
	The ability to cross a single step in the landscape is driven by the mutations available 
	to the system. More mutations of larger effect allow individuals migrating across this 
	step to adapt to the new landscape optimum and spread across the remainder of the landscape. 
	This result is similar to that of Gomulkiewicz (citation) where instead a 2-patch model 
	is simulated, and the second patch switches from a sink to a source when… ~~~~~ .
	Interestingly however, the ability of a parameter set to cross a single step does not 
	guarantee its ability to cross multiple steps. In these cases, the frequency of steps 
	occurring over the landscape impacts the ability of populations to expand and adapt.



%\section{Acknowledgements}
%We would like to thank Fredéric Guillaume for help with Nemo and members of the Whitlock lab 
%for feedback on the project. …


%%% Local Variables:
%%% TeX-master: "thesis"
%%% TeX-PDF-mode: t
%%% End:
