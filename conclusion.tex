\chapter{Conclusion}
\label{chap:conclusions}

The field of population genetics has grown by leaps and bounds since its advent. The rapid advancement of genetic and genomic techniques for acquiring data from every organism imaginable is a testament to this growth, as well as a cause. The insights allowed through these advances have greatly informed all aspects of evolutionary biology today. Even though I have not presented new data from natural organisms, my thesis work, and all theoretical work, relies heavily on information collected by evolutionary biologists through the years. We gain a great deal of knowledge by creating models of the natural world where alteration of specific factors can be controlled and investigated. Models that can be based in reality using parameters we know to be accurate to biology not only improve understanding of processes, but also their likely impact on real world populations.

I have presented four quite different projects within my thesis, each building upon the bounty of existing knowledge in the field of evolutionary biology. In \Chapref{effectivepopsize}

In \Chapref{heterogeneouslandscapes}

In \Chapref{expansionload}

In \Chapref{reproducibility}

Each of these projects has its own limitations as well as its own implications. An important running factor throughout these disparate works is the immense amount of work performed, yet the small realm of evolutionary studies that are informed. This reflects the breadth of evolutionary biology and the immensity of knowledge we have yet to understand as scientists. The simulation studies in \Chapref{effectivepopsize, heterogeneouslandscapes, expansionload} each investigate a specific range of parameter sets, as computational resources are limited, and the values deemed most important and relevant were investigated. We have gained knowledge that, however, could not be obtained from analytical models that might better describe a range of important parameters for examining a given population process. In particular, \Chapref{heterogeneouslandscapes, expansionload} have used spatially explicit simulations over large scale landscapes to investigate range expansions. Such an approach is computationally intensive and has not been used in previous studies of range expansion on this scale. The biological realism introduced by this approach has let us investigate specific questions of value, especially in \Chapref{heterogeneouslandscapes} where the structure of the landscape is a key aspect under investigation. This approach has been made publicly available for others to use in future studies and will hopefully allow continued biological insight into population genetic processes occurring in large scale landscapes.

The simulated data generated from \Chapref{effectivepopsize} was designed to test biological scenarios most relevant to scenarios occurring during estimation of effective population size in natural populations. Testing all methods available at the time on this range of data was the first fair comparison method to method. As new methods for estimating effective population size have been developed, this data will continue to be valuable in further comparisons. Since publication of \Chapref{effectivepopsize}, a new method \citep{Hui:2015} has been created for which I hope to test across all scenarios already investigated in my study.



% include future directions

understanding x is important

more about why imp

more about why we didn't know this

then about what we found

implications of what we found

examples of the predicted implications maybe happening

what are the possible future directions worth investigating for each chapter

what were the coolest results from each and those implications?

taking all this theory to the real world is worthwhile


limitations of these approaches/studies