\chapter{Conclusion}
\label{chap:conclusions}

The field of population genetics has grown by leaps and bounds since its advent. The rapid advancement of genetic and genomic techniques for acquiring data from every organism imaginable is a testament to this growth, as well as a cause. The insights allowed through these advances have greatly informed all aspects of evolutionary biology today. Even though I have not presented new data from natural organisms, my thesis work, and all theoretical work, relies heavily on information collected by evolutionary biologists through the years. We gain a great deal of knowledge by creating models of the natural world where alteration of specific factors can be controlled and investigated. Models that can be based in reality using parameters we know to be accurate to biology not only improve understanding of processes, but also their likely impact on real world populations.

I have presented four quite different projects within my thesis, each building upon the bounty of existing knowledge in the field of evolutionary biology. In \Chapref{effectivepopsize}, I tested the accuracy of methods designed for estimating effective population sizes. Out of a range of existing studies, I found that two methods perform best. I also found that their best performance varied across different demographic scenarios of migration among populations. This lends credit to the proposition made throughout my thesis, that knowing the demographic history of populations is both useful and necessary to make proper evolutionary inferences from data.

In \Chapref{heterogeneouslandscapes}, I simulated cases of range expansions over environmental gradients that varied over space in terms of patchiness and steepness. Varying the genetic architecture of the trait underlying adaptation to the environment showed different abilities of populations to adapt and expand their ranges. Two qualitatively different regimes resulted whereby expansion could succeed from either many small effect alleles or few large effect alleles. Intermediate between these two regimes, expansion was more limited, indicating that this may be the parameter space for which adaptation is most difficulty. Though the upper bound of the mutation rates included may be unrealistic, this result provides impetus for future investigation of the interaction between mutation numbers, effect sizes, and the scale of change in the environment.

In \Chapref{expansionload}, we examined range expansions with combinations of conditions generating either or both expansion load and local maladaptation to an environmental gradient. We found an interesting interaction between these two factors, whereby the fitness reduction cause by expansion load accumulating at expanding range fronts reduces population fitness. In the presence of steep environmental gradients, a fitness cost is already incurred as populations attempt to adapt to the local environment. When combined, these effects compound and lead to the extinction of populations at the range edge, instead leaving behind a different range edge or more locally adapted individuals. Because any lower degree of local adaptation would combine with expansion load to generate populations with fitness levels below those of persistence, those populations simply no longer exist, and local adaptation is improved at the range edge at a cost of slower range expansion. Investigating further parameter space around these simulation as well as investigating other biologically realistic possibilities such as evolution of dispersal provide ample future directions for this research.

In \Chapref{reproducibility}, we performed a reproducibility study on published datasets using the analysis method \textsc{structure}. We found that the same biological result of $K$ values could be reproduced in 70\% of studies. However, if we measure this success at the starting point of acquiring the datasets for which reanalyses were performed, this percentage would decrease greatly. These results are nonetheless promising, as they indicate the successful repeatability of a stochastic genetic analysis method. It would be interesting to conduct a similar reproducibility study today, since the standards for data archiving have greatly improved in the four years since publication of this study.

Each of these projects has its own limitations as well as its own implications. An important running factor throughout these disparate works is the immense amount of work performed, yet the small realm of evolutionary studies that are informed. This reflects the breadth of evolutionary biology and the immensity of knowledge we have yet to understand as scientists. The simulation studies in \Chapref{effectivepopsize, heterogeneouslandscapes, expansionload} each investigate a specific range of parameter sets, as computational resources are limited, and the values deemed most important and relevant were investigated. We have gained knowledge that, however, could not be obtained from analytical models that might better describe a range of important parameters for examining a given population process. In particular, \Chapref{heterogeneouslandscapes, expansionload} have used spatially explicit simulations over large scale landscapes to investigate range expansions. Such an approach is computationally intensive and has not been used in previous studies of range expansion on this scale. The biological realism introduced by this approach has let us investigate specific questions of value, especially in \Chapref{heterogeneouslandscapes} where the structure of the landscape is a key aspect under investigation. This approach has been made publicly available for others to use in future studies and will hopefully allow continued biological insight into population genetic processes occurring in large scale landscapes.

The simulated data generated from \Chapref{effectivepopsize} was designed to test biological scenarios most relevant to scenarios occurring during estimation of effective population size in natural populations. Testing all methods available at the time on this range of data was the first fair comparison method to method. As new methods for estimating effective population size have been developed, this data will continue to be valuable in further comparisons. Since publication of \Chapref{effectivepopsize}, a new method \citep{Hui:2015} has also been created for which I hope to test across all scenarios already investigated in my study. A new study has also performed a comparison across a subset of these methods to test their performance in a different range of biological scenarios as well as with differing effects of genetic markers on these inferences \citep{Wang:2016}. This is a trend that may be best seen for methods estimating effective population size. Several previous studies have presented assessments of method performance \citep{Ryman:2013, Neel:2013, Holleley:2013, Hoehn:2012, Barker:2011}, and author who develop these methods generally present an evaluation of their approach upon publication. This method testing is vital, but often lacks consistency in comparison across methods and datasets, resulting in a comparison of apples to oranges. Methods testing can be difficult and time-consuming, however, and no author is generally able to predict all situations for which their method may be applied. The burden thus rests on researchers applying these methods to ensure that they are meeting the appropriate assumptions of the method, or to first test how applying a method may be biased in their biological scenario.

A great effort which has yet to come to fruition, is the idea to maintain an online database of simulated data for methods testing. This would standardize comparisons from different researchers to the same test dataset. Simulated data allows accurate assessment of analysis results because the true parameters or biological answers within the simulated data are known. This would also increase the ease of methods testing, as authors do not have to continually generate their own test datasets. Furthermore, researchers applying a dataset to a new scenario that may be assumption-violating would be able to find equivalent simulated data and test the method themselves in order to understand how useful it may be. The finalization of this sort of database would be a great boon to advancing population genetic analyses, particularly as more and more genetic analyses methods are formulated. As more data, simulated and natural, continues to be archived at publication and resources for long-term storage and access to data grow, the state of data availability is optimistic for evolutionary biology.

With the relative ease of acquiring genetic data, population genetics is beginning to face a larger problem of learning new ways to analyze large genomic datasets. No longer is it the case that we lack in data; we now lack the ability to fully make use of this data in many cases. Until recently, the majority of existing methods functioned only for small sets of loci. Genomic advances now provide researchers with abundances of data that can overload analysis methods. A through understanding of the value gained from additional loci past a certain point is lacking, but if we operate under the assumption that more data is always better, then valuable information may be lost when methods can not accommodate all of the data. A temporary solution used by some researchers in applying a method designed for few loci is simply subset their data. Subsetting is often a non-optimal solution, because multiple analyses of different subsets are then still necessary to ensure no bias has unknowingly entered the subset data. Fortunately, more and more methods continue to be developed for handling genomic data on large scales; for example analyses of population structure have greatly advanced \citep{Raj:2014,Bradburd:2016, Petkova:2015}.

Major questions still remain to be answered in evolutionary biology. Particularly relevant to my projects on adaptation during range expansion and under differing genetic architectures, as well as to the topic of large scale data analysis, is the question of how many loci contribute to local adaptation. Detecting small effect loci in real datasets is extremely difficult, compounded by the statistical problem of comparing across huge sets of loci. Through these simulation studies, I hope that I have contributed to interesting avenues of research that will continue to be investigated. From \Chapref{expansionload}, we saw that the effects of multiple loci on fitness and adaptation did not combine multiplicatively, but instead interacted. This both lends credit to as well as points out the faults in learning biology from simplified models. Our model clearly improved upon any using simply one locus. Were one to predict the effects on population survival during range expansion by extrapolating from a one-locus model, the result would be highly unrealistic, as adding more loci creates the potential for ever-increasing interactions. Such interactions are likely to be rife throughout biology, and understanding these interactions through controlled simulation experiments becomes valuable for creating a strong foundation of predictions to be tested in real world data. In combination, theoretical and biological insights become the most valuable for advancing our knowledge of evolution, and I hope that the results of my work lead to worthwhile investigations of population genetic processes in species alive today.


