\chapter{Conclusion}
\label{chap:conclusions}

The field of population genetics has grown by leaps and bounds since its advent. The rapid advancement of genetic and genomic techniques for acquiring data from every organism imaginable is a testament to this growth, as well as a cause. The insights allowed through these advances have greatly informed all aspects of evolutionary biology today. Even though I have not presented new data from natural organisms, my thesis work, and all theoretical work, relies heavily on information collected by evolutionary biologists through the years. We gain a great deal of knowledge by creating models of the natural world where alteration of specific factors can be controlled and investigated. Models that can be based in reality using parameters we know to be accurate to biology not only improve understanding of processes, but also their likely impact on real world populations.

I have presented four quite different projects within my thesis, each building upon the bounty of existing knowledge in the field of evolutionary biology. In \Chapref{effectivepopsize}

In \Chapref{heterogeneouslandscapes}

In \Chapref{expansionload}

In \Chapref{reproducibility}

Each of these projects has its own limitations as well as its own implications. An important running factor throughout these disparate works is the immense amount of work performed, yet the small realm of evolutionary studies that are informed. This reflects the breadth of evolutionary biology and the immensity of knowledge we have yet to understand as scientists. The simulation studies in \Chapref{effectivepopsize, heterogeneouslandscapes, expansionload} each investigate a specific range of parameter sets, as computational resources are limited, and the values deemed most important and relevant were investigated. We have gained knowledge that, however, could not be obtained from analytical models that might better describe a range of important parameters for examining a given population process. In particular, \Chapref{heterogeneouslandscapes, expansionload} have used spatially explicit simulations over large scale landscapes to investigate range expansions. Such an approach is computationally intensive and has not been used in previous studies of range expansion on this scale. The biological realism introduced by this approach has let us investigate specific questions of value, especially in \Chapref{heterogeneouslandscapes} where the structure of the landscape is a key aspect under investigation. This approach has been made publicly available for others to use in future studies and will hopefully allow continued biological insight into population genetic processes occurring in large scale landscapes.

The simulated data generated from \Chapref{effectivepopsize} was designed to test biological scenarios most relevant to scenarios occurring during estimation of effective population size in natural populations. Testing all methods available at the time on this range of data was the first fair comparison method to method. As new methods for estimating effective population size have been developed, this data will continue to be valuable in further comparisons. Since publication of \Chapref{effectivepopsize}, a new method \citep{Hui:2015} has also been created for which I hope to test across all scenarios already investigated in my study. A new study has also performed a comparison across a subset of these methods to test their performance in a different range of biological scenarios as well as with differing effects of genetic markers on these inferences \citep{Wang:2016}. This is a trend that may be best seen for methods estimating effective population size. Several previous studies have presented assessments of method performance \citep{Ryman:2013, Neel:2013, Holleley:2013, Hoehn:2012, Barker:2011}, and author who develop these methods generally present an evaluation of their approach upon publication. This method testing is vital, but often lacks consistency in comparison across methods and datasets, resulting in a comparison of apples to oranges. Methods testing can be difficult and time-consuming, however, and no author is generally able to predict all situations for which their method may be applied. The burden thus rests on researchers applying these methods to ensure that they are meeting the appropriate assumptions of the method, or to first test how applying a method may be biased in their biological scenario.

A great effort which has yet to come to fruition, is the idea to maintain an online database of simulated data for methods testing. This would standardize comparisons from different researchers to the same test dataset. Simulated data allows accurate assessment of analysis results because the true parameters or biological answers within the simulated data are known. This would also increase the ease of methods testing, as authors do not have to continually generate their own test datasets. Furthermore, researchers applying a dataset to a new scenario that may be assumption-violating would be able to find equivalent simulated data and test the method themselves in order to understand how useful it may be. The finalization of this sort of database would be a great boon to advancing population genetic analyses, particularly as more and more genetic analyses methods are formulated. As more data, simulated and natural, continues to be archived at publication and resources for long-term storage and access to data grow, the state of data availability is optimistic for evolutionary biology.

With the relative ease of acquiring genetic data, population genetics is beginning to face a larger problem of learning new ways to analyze large genomic datasets. No longer is it the case that we lack in data; we now lack the ability to fully make use of this data in many cases. Until recently, the majority of existing methods functioned only for small sets of loci. Genomic advances now provide researchers with abundances of data that can overload analysis methods. A through understanding of the value gained from additional loci past a certain point is lacking, but if we operate under the assumption that more data is always better, then valuable information may be lost when methods can not accommodate all of the data. A temporary solution used by some researchers in applying a method designed for few loci is simply subset their data. Subsetting is often a non-optimal solution, because multiple analyses of different subsets are then still necessary to ensure no bias has unknowingly entered the subset data. 

because to get at some of the intricate demog stuff I talked about in range exp chaps, we need the little details that may only come from large scale genomic datasets

% include future directions

understanding x is important

more about why imp

more about why we didn't know this

then about what we found

implications of what we found

examples of the predicted implications maybe happening

what are the possible future directions worth investigating for each chapter

what were the coolest results from each and those implications?

taking all this theory to the real world is worthwhile


limitations of these approaches/studies