\chapter*{Abstract}
\chaptermark{abstract}
\addcontentsline{toc}{chapter}{Abstract}

%% 344 currently
Evolution is driven by four major processes that create, maintain, or eliminate genetic diversity within and among populations: mutation, gene flow, genetic drift, and natural selection. My thesis examines the role of demographic history and its interactions with each of these processes in impacting the evolution of populations. Demographic history can cause various states of non-equilibria in populations creating the potential to mis-inform important evolutionary inferences. Such inferences may be key for making conservation decisions in applied biology. \Chapref{effectivepopsize} investigates methods for estimating effective population sizes under the assumption-violating scenario of migration among populations. Effective population size is proportional to the amount of genetic drift a population experiences, yet gene flow can affect measures of drift and thus estimates of population size. Using simulated data to understand the impact of migration on estimation accuracy, I find that two existing estimation methods function best. I next present two studies on species range expansions and the roles of migration, mutation, selection, and drift on expansion dynamics. Range expansion is a common demographic history in many species and can lead to non-equilibrium genetic scenarios. The first of these studies shows the interaction of deleterious mutation accumulation and local adaptation to environmental gradients during range expansions (\Chapref{expansionload}). The interplay of expansion load, mutation load, and migration load lead to different levels of local adaptation in expanding populations. \Chapref{heterogeneouslandscapes} examines the ability of species to expand over patches of environmental optima under different genetic architecture regimes. Expansion is enhanced by certain genetic architectures, and each of these interacts with the size of patches on the landscape as well as how strongly selection varies across patches. My final study assesses the reproducibility of analyses using the common stochastic algorithm \textsc{structure} (\Chapref{reproducibility}). This research finds $30\%$ failure of reproducibility for results from \textsc{structure} using published datasets and elucidates the reasons for failure of reproducibility. In sum, my thesis contributes to our understanding of how gene flow, population size, heterogeneous selection, and mutation interact to impact the genetics of populations and thus the fate of evolving biodiversity.

%%% Local Variables:
%%% TeX-master: "thesis"
%%% TeX-PDF-mode: t
%%% End:


%% MUST BE < 350 WORDS
%% MUST BE ON PAGE ii

%% NO ABSTRACTS WITHIN CHAPTERS

