\chapter*{Abstract}
\chaptermark{abstract}
\addcontentsline{toc}{chapter}{Abstract}

%% 362 currently
Evolution is driven by four major processes that create, maintain, or eliminate genetic diversity within and among populations: mutation, gene flow, genetic drift, and natural selection. My thesis examines the role of demographic history and its interactions with each of these processes in impacting the evolution of populations. Demographic history can cause populations to exist in various states of non-equilibria which has the potential to mis-inform important evolutionary inferences. Such inferences may be key in applied biology where conservation decisions must be made. I begin with a study on the estimation of effective population sizes under the assumption-violating scenario of migration among populations (\Chapref{effectivepopsize}). Effective population size is proportional to the amount of genetic drift a population experiences, yet the presence of gene flow can affect measures of drift and thus estimates of population size. I have tested existing estimation methods using simulated data to understand the impact of migration on estimation accuracy. I next present two studies on species range expansions and the roles of migration, mutation, selection, and drift on expansion dynamics. Demographic history of range expansion is a common occurrence in many species and can lead to non-equilibrium genetic scenarios. The first of these studies explores the interaction of fitness reductions due to deleterious mutations and due to local maladaptation to the environment during range expansions (\Chapref{expansionload}). The interplay of expansion load, mutation load, and migration load lead to different levels of local adaptation in expanding populations. \Chapref{heterogeneouslandscapes} examines the ability of species to expand over patches of environmental optima under different genetic architecture regimes. Expansion is enhanced by different forms of genetic architecture, and each of these interacts with the size of patches on the landscape as well as how strongly selection varies across patches. My final study assesses the reproducibility of analyses using the common algorithm \textsc{structure} (\Chapref{reproducibility}). This research investigates the ability to regenerate results from this stochastic analysis method using published datasets, and elucidates the reasons for cases where reproducibility failed. In sum, my thesis contributes to our understanding of how gene flow, population size, heterogeneous selection, and mutation interact to impact the genetics of populations and thus the fate of evolving biodiversity.

%%% Local Variables:
%%% TeX-master: "thesis"
%%% TeX-PDF-mode: t
%%% End:


%% MUST BE < 350 WORDS
%% MUST BE ON PAGE ii

%% NO ABSTRACTS WITHIN CHAPTERS

%% 428 words :/
%% Evolution is driven by four major processes: mutation, gene flow, genetic drift, and natural selection. Each of these plays a major role in determining the future of species and whether populations will adapt and survive or go extinct. My thesis addresses each these processes through examining the role of demographic history and its impact on the evolution of populations. Demographic history can cause populations to exist in various states of non-equilibria which has the potential to mis-inform important evolutionary inferences. Such inferences may be key in applied biology where conservation decisions must be made. I begin my investigations of this effect of demographic history with a study on the estimation of effective population sizes under the non-equilibrium, assumption-violating scenario of migration among populations (\Chapref{effectivepopsize}). Effective population size is proportional to the amount of genetic drift a population experiences, yet the presence of gene flow can affect measures of drift and thus estimates of population size. I have tested existing estimation methods using simulated data to understand the impact of migration on accuracy of estimates. I next present two studies on species range expansions and the roles of migration, mutation, and selection on expansion dynamics. Range expansions are a common occurrence in many species and a major case of non-equilibrium demographic history. The first of these studies examines the ability of species to expand over heterogeneously changing environmental optima under different genetic architecture regimes (\Chapref{heterogeneouslandscapes}). Expansion requires that edge populations adapt to new conditions, but with the added condition of colonization occurring from edge populations of smaller effective population sizes where the strength of selection is reduced. Adaptation during expansion becomes easier with larger patches of the same selective pressures, but adaptation and expansion are inhibited on very patchy landscapes if selection varies greatly across patches. The second of these studies explores the actions and interactions of fitness reductions due to deleterious mutations and due to local maladaptation to the environment on populations at the expanding edge of a species range (\Chapref{expansionload}). This interplay of expansion load, mutation load, and migration load lead to different levels of local adaptation in expanding populations. My final study assesses the reproducibility of analyses using the common algorithm \textsc{structure} (\Chapref{reproducibility}). This research investigates the ability of stochastic genetic analysis methods to recreate identical results from the same data, and elucidates the reasons for cases where reproducibility failed. In sum, my thesis contributes to our understanding of how gene flow, population size, landscape heterogeneity, and mutation interact and impact the genetics of populations and thus the fate of evolving biodiversity.

