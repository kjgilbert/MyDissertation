\chapter*{Abstract}
\chaptermark{abstract}
\addcontentsline{toc}{chapter}{Abstract}

Evolution is driven by four major processes: mutation, gene flow, genetic drift, and natural selection. Each of these plays a major role in determining the future of species and whether populations will adapt and survive or go extinct. My thesis addresses multiple questions relevant to these processes, where I focus on the role of demographic history and its potential to mis-inform evolutionary inferences in the presence of different, non-equilibrium demographic histories. I begin with a study on the estimation of effective population sizes under the non-equilibrium, assumption-violating scenario of migration among populations (\Chapref{effectivepopsize}). Effective population size is proportional to the amount of genetic drift a population experiences, yet the presence of gene flow can affect measures of drift and thus estimates of population size. I have tested existing estimation methods using simulated data to understand the impact of migration on accuracy of estimates. 
I next present two studies on species range expansions and the roles of migration, mutation, and environment on expansion dynamics. Range expansions are a common occurrence in many species and a major case of non-equilibrium demographic history. The first of these studies examines the ability of species to expand over heterogeneously changing environmental optima under different genetic architecture regimes (\Chapref{heterogeneouslandscapes}). Expansion requires that edge populations adapt to new conditions, but with the added condition of colonization occurring from edge populations of smaller effective population sizes where the strength of selection is reduced. The second of these explores the actions and interactions of fitness reductions due to deleterious mutations (mutation load) and due to local maladaptation to the environment (migration load) on populations at the expanding edge of a species range (\Chapref{expansionload}). My final study assesses the reproducibility of analyses using the common algorithm \textsc{structure} (\Chapref{reproducibility}). This research investigates the ability of stochastic analysis methods to recreate identical results from the same data, and elucidates the reasons for cases where reproduction failed. In sum, my thesis contributes to our understanding of how gene flow, population size, landscape heterogeneity, and mutation interact and impact the genetics of populations and thus the fate of biodiversity.

%%% Local Variables:
%%% TeX-master: "thesis"
%%% TeX-PDF-mode: t
%%% End:
