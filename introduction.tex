\chapter{Introduction}
\label{chap:introduction}

% Yay inspiriational quotes.
\begin{quoteshrink}
% alternate quote: "The change from one stable equilibrium to the other may take place as the result of the isolation of a small unrepresentative group of the population, a temporary change in the environment which alters the relative viability of different types, or in several other ways..."
  ``I do not propose to argue the case for evolution, which I regard as being quite as well proven as most other historical facts, but to discuss its possible causes, which are certainly debatable.''
  \hfill\citet{Haldane:1932}, p.3
\end{quoteshrink}
% A Review of Some Fundamental Concepts and Problems of Population Genetics', Cold Spring Harbor Symposia on Quantitative Biology, 1955, 20, 13-14

\noindent


% great theme to come back to is the faults in simplified models of evolution

Evolutionary biology, as a discipline, aims to understand the genesis and maintenance of the immense biodiversity in existence today. An essential part of this task requires understanding the genetic processes and patterns that occur within and among populations. Population genetics coalesced as a field in the early 1900s, following more than a decade of controversy and debate over the process of evolution \citep{Provine:2001}. Since its advent, population genetics has made great insight into the processes that contribute to the evolution of populations and species. Our understanding of genetic drift, gene flow, mutation, and selection have created a solid foundation for the growing field of population genetics. Yet as the field has advanced, realization that the manifold and minute complexities of population genetic processes can immensely impact the evolution of populations and species leaves many open questions in the field of evolutionary biology.

These questions can be addressed through theoretical studies, using analytic models or running simulations, or through empirical studies, gathering observational and experimental biological data for analysis. Both approaches to understanding evolution complement each other, and require insight from each other. Population genetic theory uses simplified models that lead to evolutionary predictions and expectations that are then tested in the real world. Biology is always more complicated, and when results in the real world differ from our theoretical predictions, the process repeats, more complex models are developed and tested in the real world, and our understanding of biology advances. In my thesis, I take a simulation approach to further our understanding of population genetics in several interesting, non-equilibrium scenarios.

% demography
The demographic history of a population plays a major role in its evolution. Demography describes the characteristics of a population's size, distribution, and composition through statistics such as birth or death rates \citep{Pavlik:2000}. This information is useful for short-term predictions in populations, but in combination with population genetic data a more complete understanding of a population's past, present, and future can be attained, proving especially vital in conservation biology \citep{Lande:1988, Avise:1995, Nunney:1993}. Effective population size serves as an example of this situation. Census size of many organisms can be obtained, as this simply requires counting all individuals present in a population (though for many reasons, census size can often be the more difficult quantity to obtain). However census size alone does not generally inform long term processes occurring in a population nor how the population may evolve. Genetic data is necessary for such insight. For example, a large number of individuals might be present at any given time due to stochastic events such as an increase in resources, rather than reflecting consistently large numbers of individuals. Likewise a small number of individuals might be found if other stochastic abiotic factors such as a recent catastrophic storm lead to the death of many individuals. In these instances, defining a population by its effective size is more useful for informing the evolutionary processes occurring within the population and can reveal information that the census size does not. 

% Ne
The effective population size is a key parameter in population genetics and evolution. Defined by \citet{Wright:1931} as the size of an ideal population that experiences the same level of genetic drift as the given population, effective population size informs how populations are likely to respond to evolutionary forces. Though it can be defined in several different ways depending on which aspect of drift, temporal scale, or other population property is of interest, Wright's definition serves best for the contemporary effective size of a local population. The effective population size has been used widely as a deciding parameter in conservation situations such as suggesting minimally viable population sizes or establishing breeding populations in both plants and animals \citep{Lande:1987, Soule:1987, Ellstrand:1993}.

% Ne and demography -> gene flow
Estimating the effective population size for these purposes in natural population can, however, be a difficult task. We use genetic data to understand the degree of genetic drift a population experiences, and thus its effective size. This is generally accomplished by examining changes in allele frequencies over time. Allele frequencies within populations over time are often, if not always, subject to change due to forces other than random genetic drift alone. Populations are subject to natural selection, mutation, and gene flow among populations -- processes that are all capable of altering allele frequencies in multitudes of ways. Methods developed for the purpose of estimating effective size attempt to work past some of these difficulties. Using datasets only of neutral loci removes the biases of selection altering allele frequencies and biasing estimation of effective population size, at least to the best of our ability in identifying which loci are truly neutral. Using data over very short times scales allows us to safely assume a lack of mutational input for the dataset at hand. However, the final process in this list, gene flow, is not as easily ignored.

Gene flow varies widely in amount across species, but is likely a prevalent force in many populations and species across the globe \citep{Slatkin:1985, Slatkin:1987}. 
Estimating gene flow and migration rates alone is already a daunting task possessing its own estimation methods and their accompanied difficulties \citep{Beerli:2001, Hey:2004, Kuhner:2006, Whitlock:1999}. Gene flow can be a major disruptive force in evolution by introducing foreign and sometimes maladaptive individuals \citep{Slatkin:1987}. Whether dispersal into a population results in successful gene flow is linked to effective size in terms of whether selection is sufficiently strong to remove any maladapted individuals introduced, or whether populations are so small that selection is weak and gene flow instead becomes a positive force acting to sustain an otherwise extinction-bound population. The latter process is often referred to as demographic rescue and can lead to evolutionary rescue whereby gene flow introduces genetic variation and increases the effective size \citep{Carlson:2014, Bell:2013}. Even within his controversial work on shifting balance theory, Sewall Wright recognized the importance of the interplay of effective population size and migration with the evolutionary process: %% maybe take this out b/c it's about shifting balance
\begin{quoteshrink}
  ``The rate of evolutionary change depends primarily on the balance between the effective size of population in the local strain and the amount of interchange of individuals with the species as a whole \dots''
  \hfill\citet{Wright:1930}
\end{quoteshrink}
This introduction of foreign alleles has the potential to alter allele frequencies on top of any levels of genetic drift a population is subject to, and disentangling the effects of these disparate processes can be difficult. The caveat introduced with the majority of estimators for effective population size is that the population must be isolated, but this assumption is likely violated often in nature. Knowing the demographic history of a population can thus help to understand how the estimation process may be biased. Without knowledge of the correct demographic history of a population, the ability of many genetic analysis methods to correctly infer a parameter of interest can be hampered. For example, if a population under study recently underwent a vicariance event and was previously part of a much larger population, the genetic patterns seen may still reflect that of the larger population. Similarly, estimating immigration rates in such a population may lead to the incorrect inference of ongoing immigration when in fact this signal is only due to the recent similarity among the previously panmictic population. Making conservation decisions based on an incorrectly overestimated effective population size in this instance could be detrimental.

In \Chapref{effectivepopsize} I investigate the impact of migration on estimation of effective population sizes through genetic data analysis. Accurate estimation of the effective population size is key in conservation biology and leads to many political decisions in terms of when and where to conserve populations at risk and how many resources to dedicate to which causes. Misinterpreting or miscalculating effective size may thus lead to the local, or even global, extinction of species in the most extreme cases. Using simulated data and testing pre-existing methods developed for estimating effective population size allows each of these methods to be equivalently tested across common demographic scenarios. Use of simulated data benefits this investigation by having knowledge of the true effective population size, which otherwise in nature is never truly possible.



% Ne and other things -> selection, drift, -> range expansions
% make this later as the link to range expansions:
Effective population size is relevant throughout many aspects of evolutionary biology for reasons other than conservation. Small populations are more subject to random genetic drift, while in larger populations selection is known to be more effective \citep{Wright:1931, Kimura:1964, Gravel:2016}. Selection more successfully acts to remove deleterious genetic material and increase the frequency of beneficial alleles in populations when the effective size is larger. Effective population size thus impacts the likelihood of new mutations establishing or being lost within a population \citep{Kimura:1962, Whitlock:2000}. Larger effective populations sizes may even lead to greater ability of local adaptation \citep{Leimu:2008}. The relationship of effective population size and natural selection is thus tightly linked, and understanding not only the current effective population size, but also demographic histories of populations thus allows us to infer the processes contributing to the genetic diversity within and among populations and species.%% ^^ MCW - this paragraph seemed out of place and perhaps unnecessary

Demographic history contributes to every population's current genetic makeup. For example, population bottlenecks, demographic growth, admixture, or migration events all leave different genetic signatures in populations. A particular demographic history, that of species range expansion, leads to interesting genetic processes and patterns in populations. When populations grow in size, they either remain geographically stationary or alternatively spread geographically across space. The latter case is a range expansion. During a species range expansion, founding populations at the expanding front exist at low effective population sizes. These populations are the source of further colonizations as the range continues to expand. Thus, the expanding front undergoes serial founder events and bottlenecks that maintain consistently low effective population sizes. Genetic drift dominates at a range edge, and not only is genetic diversity reduced, but a process termed allele surfing can occur. Drift increases the frequency of allelic variants, and these alleles proliferate across the expanding edge as the source of continued colonizations \citep{Klopfstein:2006}. This process can often be incorrectly inferred as a signal of positive selection \citep{Edmonds:2004}, but can also lead to the increased frequency of deleterious alleles that confer a mutational burden on populations and prevent their expansion or adaptation \citep{Peischl:2013, Peischl:2015, Excoffier:2009}. This accumulation of deleterious alleles is referred to as the expansion load \citep{Peischl:2013}, and has been argued as the cause for genetic diseases persisting in populations of humans since expansion out of Africa \citep{Henn:2015, Henn:2015b, Lohmueller:2008, Lohmueller:2014b, Do:2015}. Theoretical studies have simulated this process and shown that these fitness reductions can be substantial in populations at the range edge \citep{Peischl:2013, Peischl:2015, Peischl:2015b}, but the importance of this process remains somewhat debated.

% expansion load
In \Chapref{expansionload} I examine this process of expansion load during a range expansion, but introduce a new aspect of study. By combining the presence of deleterious mutations with expansion over an environmental gradient, we are able to test the interaction of expansion load with the process of local adaptation to the environment. This simulation study also explores new parameter space and incorporates a large-scale, approximately continuous landscape, providing insight into the significance and prevalence of expansion load in realistic biological scenarios. Not only does this inform predictions of how real world populations may cope with changing climate, but indicates the further importance of knowing how and why demographic history plays such an important role in species evolution.

% adaptation over patchy landscapes
In \Chapref{heterogeneouslandscapes}, I continue with models of species range expansions, but investigate a new aspect of this process. To better understand the process of local adaptation during range expansions, I incorporate patchy environmental gradients over which expansion occurs and examine the role of genetic architecture in determining this adaptive ability. The role genetic architecture plays in determining local adaptation continues to be a major topic of research \citep{Holt:2003, Yeaman:2015, Yeaman:2011, Carroll:2001, Holloway:1990, Peichel:2001, Bratteler:2006, Schiffers:2014}, and understanding how different types of genetic architecture may interact with local adaptation during range expansions may lead to valuable insight. As genomic technologies continue to advance, it is especially important to create strong theoretical predictions for how a variety of genetic architectures may behave under different demographic situations and how this may contribute to populations' abilities to adapt to changing environments. In today's world of climate change and other anthropogenic forces fragmenting or altering species' native environments, the ability of populations to adapt is key to maintaining biodiversity. %% MOE THIS LAST SENTENCE TO CONCLUSION?

Local adaptation is important in the process of evolution for many other reasons. Adapting to the local environment is what allows many species to exist across a range of conditions. In some cases, local adaptation can also be considered an early stage in the process of speciation, contributing to differentiation among populations that eventually become reproductively isolated \citep{Via:2012, Kirkpatrick:2006}. Local adaptation is a process that is highly studied in the context of the balance between gene flow and selection \citep{Kawecki:2004}, and understanding local adaptation and the maintenance of genetic diversity across populations has been a longstanding question in evolutionary biology:
\begin{quoteshrink}
  ``The roles of local fitness peaks and gene flow in adaptive evolution remain major open questions in evolutionary biology, nearly a century after Wright first raised the issue.''
  \hfill\citet{Barton:2016}
\end{quoteshrink}

Copious studies have investigated the process of range expansion and adaptation over environmental gradients \citep{Kirkpatrick:1997, Barton:2001, Bridle:2010, Polechova:2015, GarciaRamos:1997}. This theory has led to a strong understanding of the role of population sizes, genetic drift, and gene flow at range edges. An important aspect of this process is asymmetric gene flow at a range edge. Because central populations tend to be more dense under the central-marginal hypothesis \citep{Brown:1984, Eckert:2008}, proportionally more gene flow reaches populations at a range edge. This gene flow can have one of two effects: swamping locally adaptive alleles with foreign, maladaptive alleles and preventing local adaptation, or introducing necessary genetic variation in otherwise depauperate populations thereby enhancing the ability to adapt. Investigations of these processes have used linear gradients in examining this process, while many environmental gradients in the real word may change more heterogeneously over space. Whether expansion over such heterogeneous landscapes better matches theory on adaptation in two patch models \citep{Gomulkiewicz:1995, Ronce:2001, Holt:1997, Gomulkiewicz:1999}, expansion over a linear gradient, or neither necessitates my further investigations within this thesis.

Furthermore, local adaptation during range expansions is a particularly relevant investigation, as many species in the northern hemisphere have expanded and adapted northward after the last glaciation to recolonize previously uninhabitable terrain. The genetic signatures often left behind by this demographic process create difficulties in population genetic inference. A major goal of evolutionary biology today is to identify the loci that contribute to local adaptation of populations \citep{Savolainen:2013, Whitlock:2015, LeCorre:2012, Coop:2010}. Demographic history is, however, one of the leading causes of the misidentification of adaptive loci \citep{Whitlock:2015}. Range expansions in particular can lead to the spatial autocorrelation of neutral alleles with clines in environmental characteristics, leaving behind a false signal of natural selection on these loci.

%what even is a population? -> structure etc
Apart from local adaptation within populations or effective sizes of populations exists an even broader question of what a population is and how do we define populations explicitly. Understanding effective population sizes and populations undergoing range expansions necessitates that we know what a population itself is. This question is not as simple as it may sound, and determining what a true population is can be incredibly difficult in many systems, or within a given system for choosing an appropriate hierarchical level of population structure at which to define populations, or even undefinable in cases of isolation by distance \citep{Waples:2006}. Numerous methods exist and continue to be developed for the sole purpose of identifying populations or visualizing population structure in both continuous and discrete populations \citep{Pritchard:2000, Falush:2003, Falush:2007, Rosenberg:2004, Petkova:2015, Bradburd:2016}. In my final project, \Chapref{reproducibility}, I present a reproducibility study that tests the ability for a common and stochastic population clustering algorithm to repeatedly produce the same biological result. This reproducibility study replicates published analyses, and uses the same datasets to assess how biologically sound inferences of population identification and delineation are. Understanding the accuracy of our ability to appropriately detect population structure and identify populations is a vital aspect of population genetics, as these delineations are often used as the foundation for further analyses and insights into the evolution of populations. We used one of the most widely employed clustering methods, \textsc{structure} \citep{Pritchard:2000}, and examined the potential biological and computational reasons for any lack of reproducibility. Not only was this a test of the ability to correctly infer a biological parameter, but made a substantial contribution to the field in terms of promoting data archiving and sufficient descriptions of methods by authors in order to make studies more repeatable and more reproducible. 



% broad conclusion
% first sentence kind of vague
My thesis projects combined contribute to understanding the evolution of populations and species. From testing methods to simulating increasingly realistic models, these studies aim to build upon our knowledge of how different population genetic processes can alter the path of evolution as well as to how we can best infer which processes are occurring and contributing most to observed patterns of diversity. Demographic history clearly has the ability to play a large role in evolution, and understanding its interaction with population genetic processes is critical for full comprehension of the evolutionary process and all of its ramifications for the future of biodiversity.







%% - explain how knowing more about linkage and sweeps etc also matter with demography
%% - explain interaction of demog with each
%% - explain changing demog in relation to climate change etc- will be more common, so imp to study


%% A major question has emerged in the field of evolutionary biology and population genetics over the past decade: how do natural selection and past demographic history interact to shape current genetic diversity of populations and species? Researchers have come to realize that these two processes are not separate actions on the genome, but are instead intricately linked. Selective and neutral processes can occur concurrently and can often be incredibly difficult to disentangle. Recent studies1,2,3,4 provide examples of loci under selection that contribute to adaptation, and increasing amounts of genomic data allow for an improved understanding of the forces creating, maintaining, or eliminating genetic diversity. Disentangling the respective role of selection and demography is fundamental. Identifying loci that truly contribute to adaptation and determining when selection is most effective on the genome have implications for eliminating or combatting diseases, controlling invasive species, and ensuring the survival of species in the face of climate change and other anthropogenic environmental changes.

%% Demographic history contributes to every population's current genetic makeup. For example, population bottlenecks, demographic growth, admixture, or migration events all leave different genetic signatures in populations. Methods exist to infer such histories5,6, but it has recently been discovered that some demographic histories, particularly range expansions, can leave genetic signatures identical to others, confounding our ability to infer a population's past and what has contributed to its present and future. During a species range expansion, founding populations of small size at the expanding front experience a process termed gene surfing7, which can increase the frequency of neutral variants above expectation in newly colonized habitats, and look very much like an event of positive selection8. This same process can also result in the increase of deleterious variants that confer a mutational burden on populations and prevent their expansion or adaptation9 to new conditions due to the reduced efficiency of selection in small populations at range edges. The importance of these spatial expansions on the genetic diversity of populations has only recently been identified. Furthermore, a related demographic history of a range shift, where the entirety of a species range moves over space expanding at the front and receding at the trailing edge, has not been investigated in terms of its potential effects on population diversity. Furthering our understanding of these processes and their impact on species diversity will thus be a great advancement of the fields of evolutionary biology and population genetics.

%% 1) Stapley et al 2010 TREE 25:705-712, 2) Excoffier et al 2009 Heredity 103:285-298, 3) Stinchecombe and Hoekstra 2008 Heredity 100:158-170, 4) Foll and Gaggiotti 2008 Genetics 180:977-993, 5) Ramakrishnan et al 2005 MEC 14:2915-2922, 6) Hey and Nielsen 2004 Genetics 167:747-760, 7) Klopfstein et al 2006 MBE 23:482-490, 8) Edmonds et al 2004 PNAS 101:975-979, 9) Peischl et al 2013 MEC 22:5972-5982,  10) Tishkoff et al 2007 Nature Genetics 39:31-40




%%\section{Structure and contents of this thesis}



% version including adaptree?
% Population genetics as a field is experiencing a stimulating period of research. Genomic capabilities have advanced at astonishing speeds and continue to do so, providing easy access into insights previously beyond the reach of researchers studying non-model organisms. Simultaneously, the world is facing rapid environmental changes due to causes such as climate change and anthropogenic forces that result in habitat fragmentation, invasive species, or other harmful actions on species. Combined, these two factors mean that not only are we in desperate need of understanding the processes underlying evolution in changing environments to preserve biodiversity into the future, but we are also increasingly able to accomplish exactly these goals. However, despite the increasing ease of data generation, inferential methods and analyses require analogous advances to succeed in expanding our knowledge on evolutionary processes and patterns.
% In this thesis, I present my research which has touched in part on all of these topics. I have assessed the capabilities of methods for estimating effective population size to function accurately in the presence of migration, a factor which can skew estimation results in unexpected ways due to allele frequency changes. I have performed a validation study of genetic loci identified through genome wise associations and environment-allele correlations to test for the presence of false positives. And I have performed simulation studies to elucidate the intricacies of species range expansions and the role of heterogeneously changing environmental gradients, mutation load, and migration load on preventing or enhancing range expansions.

%Population size is a defining characteristic in countless ways. Census size of populations at the broadest level can describe the density at which species exist. Yet often, even describing what a population itself is can be a difficult task. One of the most essential parameters in evolutionary biology however, is the effective population size. This measure of a population is defined by the amount of genetic drift a population experiences. It describes the genetic diversity present, and thus how likely the population is to persist into the future. Effective population size further relates to many major evolutionary processes







%%% Local Variables:
%%% TeX-master: "thesis"
%%% TeX-PDF-mode: t
%%% End:
