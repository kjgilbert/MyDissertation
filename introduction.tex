\chapter{Introduction}
\label{chap:introduction}

% Yay inspiriational quotes.
\begin{quoteshrink}
% alternate quote: "The change from one stable equilibrium to the other may take place as the result of the isolation of a small unrepresentative group of the population, a temporary change in the environment which alters the relative viability of different types, or in several other ways..."
  ``I do not propose to argue the case for evolution, which I regard as being quite as well proven as most other historical facts, but to discuss its possible causes, which are certainly debatable.''
  \hfill\citet{Haldane:1932}, p.3
\end{quoteshrink}
% A Review of Some Fundamental Concepts and Problems of Population Genetics', Cold Spring Harbor Symposia on Quantitative Biology, 1955, 20, 13-14

\noindent
%Put some content in here, and cite some people \citep{Allen-2003}.

% version including adaptree?
% Population genetics as a field is experiencing a stimulating period of research. Genomic capabilities have advanced at astonishing speeds and continue to do so, providing easy access into insights previously beyond the reach of researchers studying non-model organisms. Simultaneously, the world is facing rapid environmental changes due to causes such as climate change and anthropogenic forces that result in habitat fragmentation, invasive species, or other harmful actions on species. Combined, these two factors mean that not only are we in desperate need of understanding the processes underlying evolution in changing environments to preserve biodiversity into the future, but we are also increasingly able to accomplish exactly these goals. However, despite the increasing ease of data generation, inferential methods and analyses require analogous advances to succeed in expanding our knowledge on evolutionary processes and patterns.
% In this thesis, I present my research which has touched in part on all of these topics. I have assessed the capabilities of methods for estimating effective population size to function accurately in the presence of migration, a factor which can skew estimation results in unexpected ways due to allele frequency changes. I have performed a validation study of genetic loci identified through genome wise associations and environment-allele correlations to test for the presence of false positives. And I have performed simulation studies to elucidate the intricacies of species range expansions and the role of heterogeneously changing environmental gradients, mutation load, and migration load on preventing or enhancing range expansions.

%Population size is a defining characteristic in countless ways. Census size of populations at the broadest level can describe the density at which species exist. Yet often, even describing what a population itself is can be a difficult task. One of the most essential parameters in evolutionary biology however, is the effective population size. This measure of a population is defined by the amount of genetic drift a population experiences. It describes the genetic diversity present, and thus how likely the population is to persist into the future. Effective population size further relates to many major evolutionary processes

%% TRY ANOTHER APPROACH

% moved this to ABSTRACT!
%Evolution is driven by four major processes: mutation, gene flow, genetic drift, and natural selection. Each of these processes play a major role in determining the future of species and whether populations will survive, adapt, or go extinct. My thesis has addressed multiple questions relevant to each of these processes. I begin with a study on the estimation of effective population sizes (Chapter 2). Effective population size is proportional to the amount of genetic drift a population experiences, yet the presence of gene flow can affect measures of drift and thus estimates of population size. I have tested existing estimation methods using simulated data to understand the impact of migration on accuracy of estimates. I then discuss two studies of species range expansions and the roles of migration, mutation, and environment on expansion dynamics. The first of these examines the ability of species to expand over heterogeneously changing environmental optima (Chapter 3). Expansion requires that edge populations adapt to new conditions, but under interesting conditions of smaller effective population sizes where the strength of selection is reduced. The second of these explores the actions and interactions of fitness reductions due to deleterious mutations (mutation load) and due to immigration (migration load) on populations at the expanding edge of a species range (Chapter 4). My final study assesses the reproducibility of analyses using the common algorithm \textsc{structure} (Chapter 4). This research investigates the ability of stochastic analysis methods to recreate identical results from the same data, and elucidates the reasons for cases where reproduction failed. In sum, my thesis contributes to our understanding of how gene flow, population size, landscape heterogeneity, and mutation interact and impact the genetics of populations and thus the fate of biodiversity.

%%% Local Variables:
%%% TeX-master: "thesis"
%%% TeX-PDF-mode: t
%%% End:
