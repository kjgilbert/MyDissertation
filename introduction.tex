\chapter{Introduction}
\label{chap:introduction}

% Yay inspiriational quotes.
\begin{quoteshrink}
% alternate quote: "The change from one stable equilibrium to the other may take place as the result of the isolation of a small unrepresentative group of the population, a temporary change in the environment which alters the relative viability of different types, or in several other ways..."
  ``I do not propose to argue the case for evolution, which I regard as being quite as well proven as most other historical facts, but to discuss its possible causes, which are certainly debatable.''
  \hfill\citet{Haldane:1932}, p.3
\end{quoteshrink}
% A Review of Some Fundamental Concepts and Problems of Population Genetics', Cold Spring Harbor Symposia on Quantitative Biology, 1955, 20, 13-14

\noindent
%Put some content in here, and cite some people \citep{Allen-2003}.

% version including adaptree?
% Population genetics as a field is experiencing a stimulating period of research. Genomic capabilities have advanced at astonishing speeds and continue to do so, providing easy access into insights previously beyond the reach of researchers studying non-model organisms. Simultaneously, the world is facing rapid environmental changes due to causes such as climate change and anthropogenic forces that result in habitat fragmentation, invasive species, or other harmful actions on species. Combined, these two factors mean that not only are we in desperate need of understanding the processes underlying evolution in changing environments to preserve biodiversity into the future, but we are also increasingly able to accomplish exactly these goals. However, despite the increasing ease of data generation, inferential methods and analyses require analogous advances to succeed in expanding our knowledge on evolutionary processes and patterns.
% In this thesis, I present my research which has touched in part on all of these topics. I have assessed the capabilities of methods for estimating effective population size to function accurately in the presence of migration, a factor which can skew estimation results in unexpected ways due to allele frequency changes. I have performed a validation study of genetic loci identified through genome wise associations and environment-allele correlations to test for the presence of false positives. And I have performed simulation studies to elucidate the intricacies of species range expansions and the role of heterogeneously changing environmental gradients, mutation load, and migration load on preventing or enhancing range expansions.

%Population size is a defining characteristic in countless ways. Census size of populations at the broadest level can describe the density at which species exist. Yet often, even describing what a population itself is can be a difficult task. One of the most essential parameters in evolutionary biology however, is the effective population size. This measure of a population is defined by the amount of genetic drift a population experiences. It describes the genetic diversity present, and thus how likely the population is to persist into the future. Effective population size further relates to many major evolutionary processes

%% TRY ANOTHER APPROACH

% aim for 8-9 pages?

Population genetics coalesced as a field in the early 1900s, following more than a decade of controversy and debate over the process of evolution (cite Provine). Since its advent, great insight has been made into the processes that contribute to the evolution of populations and species. Our understandings of genetic drift, gene flow, mutation, and selection have created a solid foundation for the growing field of evolutionary biology. Yet as the field has advanced, the realization of the manifold and minute complexities of these processes that can  immensely impact the process and pattern of evolution has left open questions of how and why the world's biodiversity exists.

One of the interesting complexities in understanding population genetics is the role of demographic history. Though population genetics eponymously (FIX THAT WORD) uses genetic data both in theory and empirically to understand the process of evolution, informing such genetic data with demographic information is often, if not always, necessary. Without this combination of information, it is nearly impossible to gain a full understanding of the ancestral processes leading to the system at hand and the likely future routes of this system. Effective population size makes an exemplar description of this situation. Census size of many organisms can be obtained, as this simply requires counting all individuals present in a population (though for many reasons, census size can often be the more difficult quantity to obtain). However census size alone does not generally inform long term processes occurring in a population, for example a large number of individuals might be present at a given time due to a recent stochastic increase in resources, rather than reflecting consistently large numbers of individuals. Likewise a small number of individuals might be found if abiotic factors such as a recent catastrophic storm lead to the death of many individuals. In these instances, knowing the effective population size in these systems can reveal information that the census size does not. The effective population size....

Past changes in population sizes, movement of individuals among populations, or spread of populations geographically are all actions that affect the relative importance of genetic drift, gene flow, mutation, and selection.

-explain how knowing more about linkage and sweeps etc also matter with demography

- explain interaction of demog with each

- explain changing demog in relation to climate change etc- will be more common, so imp to study


A major question has emerged in the field of evolutionary biology and population genetics over the past decade: how do natural selection and past demographic history interact to shape current genetic diversity of populations and species? Researchers have come to realize that these two processes are not separate actions on the genome, but are instead intricately linked. Selective and neutral processes can occur concurrently and can often be incredibly difficult to disentangle. Recent studies1,2,3,4 provide examples of loci under selection that contribute to adaptation, and increasing amounts of genomic data allow for an improved understanding of the forces creating, maintaining, or eliminating genetic diversity. Disentangling the respective role of selection and demography is fundamental. Identifying loci that truly contribute to adaptation and determining when selection is most effective on the genome have implications for eliminating or combatting diseases, controlling invasive species, and ensuring the survival of species in the face of climate change and other anthropogenic environmental changes.

Demographic history contributes to every population's current genetic makeup. For example, population bottlenecks, demographic growth, admixture, or migration events all leave different genetic signatures in populations. Methods exist to infer such histories5,6, but it has recently been discovered that some demographic histories, particularly range expansions, can leave genetic signatures identical to others, confounding our ability to infer a population's past and what has contributed to its present and future. During a species range expansion, founding populations of small size at the expanding front experience a process termed gene surfing7, which can increase the frequency of neutral variants above expectation in newly colonized habitats, and look very much like an event of positive selection8. This same process can also result in the increase of deleterious variants that confer a mutational burden on populations and prevent their expansion or adaptation9 to new conditions due to the reduced efficiency of selection in small populations at range edges. The importance of these spatial expansions on the genetic diversity of populations has only recently been identified. Furthermore, a related demographic history of a range shift, where the entirety of a species range moves over space expanding at the front and receding at the trailing edge, has not been investigated in terms of its potential effects on population diversity. Furthering our understanding of these processes and their impact on species diversity will thus be a great advancement of the fields of evolutionary biology and population genetics.

1) Stapley et al 2010 TREE 25:705-712, 
2) Excoffier et al 2009 Heredity 103:285-298, 
3) Stinchecombe and Hoekstra 2008 Heredity 100:158-170, 
4) Foll and Gaggiotti 2008 Genetics 180:977-993, 
5) Ramakrishnan et al 2005 MEC 14:2915-2922, 
6) Hey and Nielsen 2004 Genetics 167:747-760, 
7) Klopfstein et al 2006 MBE 23:482-490, 
8) Edmonds et al 2004 PNAS 101:975-979, 
9) Peischl et al 2013 MEC 22:5972-5982, 
10) Tishkoff et al 2007 Nature Genetics 39:31-40



good quote!! on the topic of how evolution might proceed in his view (wright's)
``A much more favorable condition [for evolution] would be that of a large population, broken up into imperfectly isolated local strains. ... The rate of evolutionary change depends primarily on the balance between the effective size of population in the local strain and the amount of interchange of individuals with the species as a whole and is therefore not limited by mutation rates.`` Wright 1930 (p166 Provine)
Ibid pp 354-55  -- becomes shifting balance theory

CAN CITE NICK BARTON'S OPINION PIECE ON SEWALL WRIGHT:
Sewall Wright on Evolution in Mendelian Populations and the "Shifting Balance"
Nicholas H. Barton
GENETICS January 5, 2016 vol. 202 no. 1 3-4; DOI: 10.1534/genetics.115.184796

"The roles of local fitness peaks and gene flow in adaptive
evolution remain major open questions in evolutionary biology,
nearly a century after Wright first raised the issue."

\section{Structure and contents of this thesis}

The aim of my thesis is to investigate a set of these processes in detail.  ... 
The first research chapter...

These projects combined contribute to our understanding of how populations and species evolve 
and how we can best measure and understand their evolution.

%%% Local Variables:
%%% TeX-master: "thesis"
%%% TeX-PDF-mode: t
%%% End:
