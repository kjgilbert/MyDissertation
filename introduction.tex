\chapter{Introduction}
\label{chap:introduction}

% Yay inspiriational quotes.
\begin{quoteshrink}
  ``The foundations of population genetics were laid chiefly by mathematical deduction from basic premises contained in the works of Mendel and Morgan and their followers. Haldane, Wright, and Fisher are the pioneers of population genetics whose main research equipment was paper and ink rather than microscopes, experimental fields, Drosophila bottles, or mouse cages. Theirs is theoretical biology at its best, and it has provided a guiding light for rigorous quantitative experimentation and observation.''
  \hfill\citet{}Theodosius Dobzhansky-1955-1 FILL IN THE CITATION AFTER BIBTEX FILE IS SET UP
\end{quoteshrink}
% A Review of Some Fundamental Concepts and Problems of Population Genetics', Cold Spring Harbor Symposia on Quantitative Biology, 1955, 20, 13-14

\noindent
%Put some content in here, and cite some people \citep{Allen-2003}.

% version including adaptree?
% Population genetics as a field is experiencing a stimulating period of research. Genomic capabilities have advanced at astonishing speeds and continue to do so, providing easy access into insights previously beyond the reach of researchers studying non-model organisms. Simultaneously, the world is facing rapid environmental changes due to causes such as climate change and anthropogenic forces that result in habitat fragmentation, invasive species, or other harmful actions on species. Combined, these two factors mean that not only are we in desperate need of understanding the processes underlying evolution in changing environments to preserve biodiversity into the future, but we are also increasingly able to accomplish exactly these goals. However, despite the increasing ease of data generation, inferential methods and analyses require analogous advances to succeed in expanding our knowledge on evolutionary processes and patterns.
% In this thesis, I present my research which has touched in part on all of these topics. I have assessed the capabilites of methods for estimating effective population size to function accurately in the presence of migration, a factor which can skew estimation results in unexpected ways due to allele frequency changes. I have performed a validation study of genetic loci identified through genome wise associations and environment-allele correlations to test for the presence of false positives. And I have performed simulation studies to elucidate the intricacies of species range expansions and the role of heterogeneously changing environmental gradients, mutation load, and migration load on preventing or enhancing range expansions.

Population size is a defining characteristic in countless ways.
 Census size of populations at the broadest level can describe the density
 at which species exist.
 More essential however is the effective population size.
 This measure of a population describes the
%%% Local Variables:
%%% TeX-master: "thesis"
%%% TeX-PDF-mode: t
%%% End:
