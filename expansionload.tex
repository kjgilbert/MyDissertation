\chapter{Range expansions over heterogeneous landscapes: the interaction of expansion load and migration load}%Mutation load, migration load, expansion load}
\label{chap:expansionload}

\section{Abstract}

\section{temporary section - our questions}

\begin{itemize}

\item How much expansion load should we expect with reasonable biological parameters, even with hard selection
Do we see as much as Peischl et al? If not, why? mutations, disperal, density?
Can we make the case that exp. load is not a problem except at the extreme edge?

are there more deleterious homozygotes in recently colonized pops?

\item How do environmental gradients and expansion load interact to determine rate of range expansion?
is rate of expansion affected by env gradient (yes)
does load ever create a range limit (no)
do adaptation to env gradients and expansion load interact to determine rate of range expansion? (seems yes)

(Note:  for these questions we want to separate the effects of range-wide mutation load from increased load 
caused by drift at the range margin.)

Do for just the cases we had + a case we find where there’s more expansion load alone
	Anova to compare the interactions
	
\item Does one type of load affect the amount of other types of load?

is migration load changed by expansion load / mutation load?
is expansion load changed by migration load?
	response variable is either type of load, based on the amount of the other type of load present
	
we can hypothesize that a steeper gradient slows expansion, so less load
we can alternatively hypothesize that a steeper gradient means a smaller Ne at the edge causing more drift and more load
	second is not true b/c inds die, so first seems to be the case
	
calculate the avg deleterious fitness at our comparison points

\item if there are interactions in the types of load, why? what is the mechanism?
if one type slows the expansion, the other benefits
with fecundity-adjusted runs - what happens?
more mut load overall is known to make everything worse and slower spread and lower Ne
because density is reduced by both, both types of load may become worse
when do individuals at the range edge hae significantly lower fitness than those at the range core? (with gradient and/or mutation load)

What combination of gradient steepness and expansion load yields the highest/lowest fitness at the range edge vs. core? 
	Total fitness edge vs core (not split delet and quanti)

How long does reduced edge fitness persist post colonization?
	? not sure if we want to look at this? Rate of recovery per 18 scenarios?

Does an environmental gradient increase edge fitness, by limiting the pace of range expansion and therefore the degree of expansion load?
If we could get at it easily, a really nice extension of this question would be: 

\item spatial patterns in load?

\item deleterious mutations
	is load due to different effect sizes?
	does surfing happen due to 1 or many loci?
	and what are the effect sizes of surfing loci?

\end{itemize}


\section{Introduction}

Species range expansions are a complex process that is widely studied in evolutionary biology. Because 
many factors can contribute to range edges and the ability or failure of a population to expand, it is 
difficult to predict species abilities to expand and adapt into novel environments. Many factors both ecological 
and evolutionary are known to contribute to range expansions, such as species competition (cite Case and Taper and other empirical papers?), 
dispersal limitations (cite?), \emph{other examples??}. An rapidly growing field of research in this area has focused however 
on the contributions of expansion load to range expansions and the resulting state of recently expanded populations (cite lots - Peischl...). 
We here investigate two major factors contributing to species range expansions: migration load and expansion load and their potential to interact.

Range expansions create an intriguing evolutionary situation for evolutionary biology. The demography 
of a range expansion is one in which serial founder events of new colonizing populations at the 
range front result in persistent reduced effective population sizes at the front. These reductions in 
\emph{N}$_e$ reduce the efficacy of selection and increase the strength of random genetic drift. 
As this continues within the expanding wave front, mutations that arise are more likely to drift, 
resulting in a process termed gene surfing wherein mutations of much larger effect sizes escape 
selection to which they would otherwise be subject and have the potential to fix or be lost in 
the edge population \citep{Klopfstein:2006}. This exact process has been posited as a reason for the 
accumulation of deleterious alleles in populations having undergone recent geographic expansions, 
in particular for studies of human diseases in populations of Quebec and Scandinavia (cite papers).

Peischl and Excoffier (cite) have described expansion load as the reduction in fitness of range edge populations 
due to gene surfing. \emph{elaborate more...}

Previous studies have also examined the impact of environmental gradients on the ability of species ranges to expand. 
A foundational study by \citet{Kirkpatrick:1997} showed that steeper environmental gradients reduce the ability 
of expansion due to failure to locally adapt as a result of migration load. For a population existing within a 
species range and on a linear environmental gradient, symmetric migration from either adjacent population 
will contribute alleles locally maladaptive, yet this symmetry on weaker gradients still allows local adaptation 
to succeed. Whereas at a range edge, immigration is asymmetric causing edge populations to diverge from their 
local optimum in the direction of immigrant individuals. When migration rates are high enough, the edge population fails 
to adapt and when the environmental gradient is steep enough goes extinct shrinking the entire species range to eventual 
extinction as asymmetric migration at the edge continues. Apart from asymmetric migration, extreme steep 
environmental gradients alone can result in too great a migration load that again the species goes extinct.
 Further studies (cite more) have improved upon this result adding additional biological realism 
 to models of range expansion... yet strong evidence of evolutionary factors causing a distinct range edge has 
 not been found.

The combination of these two effects, migration load from underlying environmental gradients and 
expansion load from genetic drift of deleterious mutations at range edges, has not previously been studied. 
We investigate here the role of each of these factors in isolation as well as in combination. 
Using individual-based simulations on a two dimensional, spatially explicit, and approximately continuous 
landscape, we compare the reductions in fitness (load) that populations experience during range expansions 
over a range of environmental gradients and with a range of deleterious mutational parameters. 
Under our range of biologically realistic parameters, we comment on the implications our results have 
in terms of understanding the presence and persistence of genetic disease in humans. \emph{remove that last sentence, or move?}



\section{Methods}

\subsection{Model}

We model diploid, monoecious individuals undergoing a range expansion over a two dimensional 
landscape grid using the program \textsc{nemo} \citep{Guillaume:2006}. Simulations were initiated 
in the left-most 40$\times$40 cells of the landscape and after burn-in to mutation-selection 
equilibrium were then allowed to expand onto a remaining 1,960$\times$40 cells of the landscape. 
Each cell on this grid had a maximum carrying capacity of 6 individuals. The simulation 
software \textsc{nemo} was modified to incorporate a breeding window, increasing local effective 
population size above this carrying capacity \emph{anyone want to do a calculation here for that Ne?}. 

Life cycle stages occur in the order of breeding, dispersal, selection, regulation. The breeding window 
enabled every individual to search for a mate in their own and surrounding cells based on a Gaussian 
probability kernel. This kernel is parameterized by $\sigma_{breed} = 0.5$ cell widths, where an individual's 
focal cell is the most likely source of a mate and the 12 surrounding cells can be searched for a mate 
with decreasing probability. When an individual fails to find a mate, it selfs, an event which most 
often happens at the expanding edge and mimics the ability of many plant species to self under 
conditions of pollen limitation (citations).

Dispersal also occurs according to a specified dispersal kernel. We tested two types of dispersal 
kernels: a Gaussian kernel and a leptokurtic kernel to test the effects of rare, long-distance dispersal 
events. $\sigma_{disperse}$ for the Gaussian dispersal kernel was set to equal 2 cell widths and the maximum dispersal 
distance was capped at $8 \times \sigma_{disperse}$. The leptokurtic dispersal kernel was created by summing 
two different Gaussian kernels with different weightings and calculated to have the same average 
dispersal distance as our Gaussian kernel: kernel one had a weighting of 0.89 and a $\sigma_{disperse}$ of 1.5 
while kernel two had a weighting of 0.11 and a $\sigma_{disperse}$ of 5.5 to create a total kurtosis of 10 for this 
distribution. The kurtotic kernel was cut off to have the same maximum dispersal distance possible, which in 
both cases meant an individual could at most disperse 16 cells in one direction from its natal cell. 
R code for calculating and discretizing the dispersal and breeding kernels is available in a 
wrapper function written for \textsc{nemo}, \textsc{anemone}, available at: \url{https://github.com/kjgilbert/aNEMOne}.

Individuals are characterized by both a quantitative trait underlain by 100 continuous loci and 1,000 deleterious loci. 
These loci were placed randomly on a genetic map of 10 chromosomes of 100cM each for each simulation replicate. 
The quantitative trait was under stabilizing selection, with $W(z) = exp[-\frac{1}{2}(z - \theta)^T\omega^{-1}(z-\theta)$ 
where $\theta$ is the environmental optimum, and $\omega$ is the ``variance-covariance matrix of selection describing the 
individual fitness surface``, according to \textsc{nemo}. The mutation rate for quantitative loci was 
set to 0.0001 and the variance in effect size per mutation to 0.02. Environmental variance was set to 1. 

Genomic mutation rates for deleterious loci was varied from 0 to 0.1 to 1.0, which for our cases of 1,000 loci 
were 0, $5\times10^{-5}$, and $5\times10^{-4}$, with back-mutation rates of 0, $5\times10^{-7}$, and $5\times10^{-6}$

\emph{is it relevant to say the quanti traits were initiated at the mean landscape optimum of the burnin area, since they're not there once the burnin finishes?}

We compared three underlying environmental gradients, all of which change linearly only along the axis of 
expansion from one end of the landscape to the other. All parameter values and their combinations are 
listed in table \ref{tab:paramresults}. Gradient values, \emph{b}, of 0 (no gradient), 0.0375, 
and 0.375 are cases where an average dispersal distance for an individual would result in a reduction 
in fitness of 0\%, 0.5\%, or 5\%.




 deleterious loci
delet.mut <- "0.0005"
gamma.mean <- 0.01
gamma.shape <- 0.3
del.dom.mean <- 0.3

 seln.model="(direct, gaussian)", seln.fitness.model="absolute",
	seln.var=variance.seln, seln.trait.dim=1, seln.local.optima=landscape, quanti.init=landscape.init.optima, num.quanti.traits=1,
	num.quanti.loci=100, quanti.mut.rate= quanti.mut, quanti.mut.var= quant.mut.var, quanti.recomb.rate= "{{100,100,100,100,100,100,100,100,100,100}}",
	quanti.init.model=1, quanti.env.var=q.env.var, num.delet.loci=1000, delet.mut.rate=delet.mut,
	delet.mut.model=1, delet.recomb.rate="{{100,100,100,100,100,100,100,100,100,100}}",
	delet.init.freq=0, delet.effects.dist="gamma", delet.effects.mean=gamma.mean, delet.effects.dist.param1=gamma.shape,
	delet.effects.dist.param2=NULL, delet.dominance.mean=del.dom.mean,
	delet.fitness.model=1, s

fecundity = 7 (except in reduced U0 cases)

2 dispersal regimes - normal/Gaussian, and kurtotic - both cut off at same limit of maximum distance dispersed

quanti mut rate = 0.0001
variance in quanti mut effect size = 0.02
Vs = 7.5
Ve = 1


Expansion load paper: 1000 deleterious alleles have a mutation rate of 0.005 with effect sizes 
following a gamma distribution of mean 0.01 and shape parameter (k ??) 0.3, and dominance of 0.3. 
Mutation happen at random locations in the genome, and a back-mutation rate of 10 to the -2 (0.00005) 
was added to Nemo’s functionality. All 1,100 loci were placed randomly onto a genetic map 
consisting of ten 100cM chromosomes for each simulation.

Mutations 
General
VG is the genetic variance in a trait.
VE is the micro-environmental variance in a trait.
VP = VG + VE is the phenotypic variance in a trait.
h2 = VG/VP is the heritability of a trait.
We will use VE to scale various parameters to one another, using VE = 1 (arbitrary). This puts everything in units of VE.
 
Quantitative Trait
Call the trait z. θz is the optimal trait value. Higher or lower trait values will reduce fitness 
(stabilizing selection). The fitness associated with a value of z is given by
wz = exp[–(z– θz)2/(2VS)] 
where VS is the inverse strength of stabilizing selection.
 
There is evidence that VS/VP ≈ 5 on average (see Kingsolver et al. 2001; Johnson and Barton 2005). 
We can frame this in terms of h2 and VE: using the general relationships above, we know that VE = 
VP – h2VP = VP­(1–h2), so VP = VE/(1–h2). Therefore VS = 5VE/(1–h2). E.g., if heritability is ⅓ and 
VE = 1, then VS = 7.5.
 
How will new mutations affect the quantitative trait? Assume a normal distribution of effects with 
mean 0. Consider the rate of increase in trait variance due to one generation of mutation: VM = 2nμE[α2], 
where 2n is the number of (diploid) loci, μ is the mutation rate per locus per generation, and E[α2] is 
the expected squared mutational effect on the trait. (Since E[α] = 0, E[α2] = V[α], the variance of mutational effects).

Empirical estimates of VM/VE (the “mutational heritability”) for various traits are on the order of 
1 × 10-3 (and probably no larger than 1 × 10-2) (see Houle et al. 1996). Assuming VE = 1, we have 10-3 = 2nμV[α]. 
We’d like to arrive at a value for 2nμ that gives a reasonable value of V[α]. We can be reasonably 
confident that the mutation rate to alleles affecting fitness is on the order of 1 per diploid genome 
per generation at least in some organisms (see deleterious mutations section). Only a fraction of those 
mutations will affect any particular trait. As a starting point, let’s imagine that only 1\% 
of the mutations that affect fitness also affect the trait. This would give 2nμ = 0.01, and V[α] = 0.1. 
Note that V[α] < VS, meaning that most mutations will generally have small effects on fitness, i.e ~ 1–5\%.
 
(There is some theory that predicts genetic variance under stabilizing selection as VG = 4UzVS, where Uz 
is the genome-wide mutation rate for the trait Uz = 2nμ (Charlesworth and Charlesworth, 2012, p. 190). 
If VS/VP ≈ 5, this means that Uz ≈ h2/20. If heritability is ⅓, Uz is about 0.017, consistent with the 
ad hoc approach above.)
 
We also need to decide on n, the number of loci that affect the trait. Using Uz = 0.01, the mutation 
rate per haploid locus will be 0.01/2n. As a starting point, 100 loci seems reasonable (e.g., 0.5\% of 20000 genes).


Fitness (deleterious mutations)
We will probably have ~ 1000 loci that can mutate from a wild-type allele to a deleterious allele, 
with no back mutation. A mutation can occur at a site that has already mutated (optional), in which 
case the site remains mutated. Mutations at all sites will be drawn from the same distribution.
 
A reasonable starting point for the deleterious mutation rate per diploid genome is Udel = 1. This 
is probably an underestimate for humans (Keightley 2012), about right for Drosophila, (Haag-Liautard et al. 2007), 
C. elegans (Denver et al. 2004) and possibly non-human endothermic vertebrates (Baer et al. 2007), and 
probably an overestimate for many other organisms (Baer et al. 2007, Halligan and Keightley 2009).
 
Most deleterious mutational effects (s) will come from a gamma distribution with two parameters, 
shape a and scale b. The mean is ab and the variance is ab2. (In Nemo, this would be parameterized 
using the mean and shape).
 
There is evidence that a < 1, leading to a leptokurtic distribution (Eyre-Walker and Keightley 2007). 
In other words, most mutations have small effects on fitness, and rare mutations have large effects 
on fitness. The mean mutational effect, ab, should not be larger than a few percent.
 
Try something like this in R, and play with the shape and scale parameters:
 
plot(dgamma(seq(0,1,0.01),shape=0.5,scale=0.1)~seq(0,1,0.01),type="l");abline(h=0)
 
We could use numbers from analyses mutation accumulation data, such as the following (highly 
leptokurtic) example (Keightley 1998), but note that these gamma parameters are highly uncertain 
among species and experiments (see Halligan and Keightley 2009).

CURRENTLY TESTING THIS GAMMA WITH TRUNCATION:
plot(dgamma(seq(0,1,0.01),shape=0.2,scale=0.037)~seq(0,1,0.01),type="l");abline(h=0) 
(comes from mean 0.01)

for 30\% below truncated cutoff: shape 0.2, mean 0.01 (avg effect, = shape*scale), scale = 0.05 
for 20\% below truncate value was shape 0.3, mean 0.01, (scale 0.03)

We may want to consider distributions (based on empirical data) that allow for a higher frequency 
of highly deleterious mutations, e.g.:

plot(dgamma(seq(0,1,0.01),shape=0.0057,scale=3.5)~seq(0,1,0.01),type="l");abline(h=0)

Flag *** our recessive lethals have 0 heterozygote effect which is unrealistic***
We will assume (as Nemo does) that h is directly proportional to s, with h = exp(-ks)/2, where 
k is a scaling factor that relates average dominance to average selection as k = –log(2E[h])/E[s]. 
If the average mutation is partially recessive, i.e., E[h] = 0.3, this would suggest k = 10. 
With the gamma parameters above, this leads to E[hs] < 1%.
 
The true distribution of mutational fitness effects is unclear, but it is likely to be more complex than a gamma distribution. 
We would also like to incorporate some fraction of new mutations, p(lethal), which have s = 1 and h = 0, i.e. recessive lethal, 
since this type of mutation is thought to be common (i.e. more common than expected from the gamma distribution above, producing 
a bimodal distribution of s; p(lethal) is probably 1–3\% per gamete in Drosophila). This will require modifications to Nemo’s code.

One modification to nemo is truncating the gamma distribution: truncated b/c want to limit the very neutral alleles that don’t 
contribute to changes in fitness, truncation happens at mu/10, created a new mean ~.015.


The other change is that we have added back mutation for deleterious mutations at mu*.01 (though .001 might be a realistic alternative we could try).


Recombination


Have decided to randomly distribute loci across the genome. This allows for some loci to be more or less linked than 
others and has the advantage of not being too specific to any one system, nor too unrealistic and therefore will 
hopefully not cause any issues with reviewers.

Across 10 chromosomes, 100 centimorgans each
	100 QTL
	1000 deleterious loci

Each replicate would have a new, random distribution.


Environmental Gradient

need 2 things:
phenotypic standard deviation
distance (sigma, from dispersal distance)

using a phenotype per latitude is complex because there are underlying genetic differences - could use some of 
Tongli’s data from Illingworth trials, or other transplant data?

want units of Vp (phenotypic standard deviations)
from K\&B: b* = sigma/sqrt(p) * b ~= 0.25 (in Garcia-Ramos and Kirkpatrick)

can logically come up with any explanation for steepness of a gradient
	want a control of 0
	test 0.1
	Garcia-Ramos and Kirkpatrick suggest 0.25
	
want to vary Vp (includes mutation?)
steepness of gradient (b*)
sigma
fecundity/density - because relates to how much drift there is

not straightforward if sigma and b can scale equivalently because the K matters in terms of how many individuals are encountered

could pick a constant b* and see what b is produced with a certain set of sigmas
maybe allow sigma to be just 3 different values for short-range dispersal, mid-range dispersal, and long-range dispersal?


\subsection{Analysis}


\section{Results}

\begin{table}[]
\centering \footnotesize
\caption{Fitness and expansion speed results per scenario parameter combinations.}
\label{tab:paramresults}
\begin{tabular}{p{0.15\textwidth}|p{0.125\textwidth}|p{0.12\textwidth}|p{0.12\textwidth}|p{0.075\textwidth}|p{0.15\textwidth}|p{0.11\textwidth}}
Environmental Gradient (\emph{b}) & Genomic Deleterious & Dispersal Kernel & Mean Expansion & Total $\bar{\omega}$	   & Quantitative Trait $\bar{\omega}$ & Deleterious Loci $\bar{\omega}$ \\
								  & Mutation Rate (U) 	&				   & Speed 			& \scriptsize{Core / Edge} & \scriptsize{Core / Edge} 		   & \scriptsize{Core / Edge} \\ \hline \hline
0.0                        & 0.0                                   & Gaussian         &                      &                            &                                         &                                       \\
0.0375                     & 0.0                                   & Gaussian         &                      &                            &                                         &                                       \\
0.375                      & 0.0                                   & Gaussian         &                      &                            &                                         &                                       \\ \hline
0.0                        & 0.1                                   & Gaussian         &                      &                            &                                         &                                       \\
0.0375                     & 0.1                                   & Gaussian         &                      &                            &                                         &                                       \\
0.375                      & 0.1                                   & Gaussian         &                      &                            &                                         &                                       \\ \hline
0.0                        & 1.0                                   & Gaussian         &                      &                            &                                         &                                       \\
0.0375                     & 1.0                                   & Gaussian         &                      &                            &                                         &                                       \\
0.375                      & 1.0                                   & Gaussian         &                      &                            &                                         &                                       \\ \hline
0.0                        & 0.0                                   & Leptokurtic      &                      &                            &                                         &                                       \\
0.0375                     & 0.0                                   & Leptokurtic      &                      &                            &                                         &                                       \\
0.375                      & 0.0                                   & Leptokurtic      &                      &                            &                                         &                                       \\ \hline
0.0                        & 0.1                                   & Leptokurtic      &                      &                            &                                         &                                       \\
0.0375                     & 0.1                                   & Leptokurtic      &                      &                            &                                         &                                       \\
0.375                      & 0.1                                   & Leptokurtic      &                      &                            &                                         &                                       \\ \hline
0.0                        & 1.0                                   & Leptokurtic      &                      &                            &                                         &                                       \\
0.0375                     & 1.0                                   & Leptokurtic      &                      &                            &                                         &                                       \\
0.375                      & 1.0                                   & Leptokurtic      &                      &                            &                                         &                                       \\ \hline
\end{tabular}
\end{table}

\section{Discussion}


human papers get different answers - because dominance assumptions matter

our results show that we can detect X amount of load due to loci of greater than or equal to effect size of Y

\subsection{Future Directions}
Can we predict when edge vs. central individuals are more fit at and beyond the current range limit?  
Would require a different aspect of the simulation so maybe it’s not worth it for this paper, but would tie our results directly to prominent ecological/conservation issues with range expansion, e.g. are edge populations worth conserving? Which genotypes should one use for assisted migration? 
	Could be a future direction

evolution of dispersal


%%% Local Variables:
%%% TeX-master: "thesis"
%%% TeX-PDF-mode: t
%%% End:
